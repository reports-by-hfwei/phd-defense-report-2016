\tdplotsetmaincoords{60}{135}

\begin{tikzpicture}[font = \scriptsize, framed, background rectangle/.style = {draw = teal}, scale = 2.6, tdplot_main_coords]
\coordinate (O) at (0,0,0);
\coordinate (x) at (1,0,0);
\coordinate (y) at (0,1,0);
\coordinate (z) at (0,0,1);

\uncover<1->{
\node[font = \normalsize, rotate = -90, teal] at (0,1.4,1.4) {\textsc{Datatype}};
}

\uncover<2->{
  \draw[thick, >=Stealth, ->] (O) to node[sloped, below, near end, align = center] {\texttt{\rmfamily Mechanism}\\{\tiny (\texttt{机制: 怎么做})}} (x);
  \draw[thick, >=Stealth, ->] (O) to node[anchor = north, sloped, below, near end, align = center] {\texttt{\rmfamily Measurement}\\{\tiny (\texttt{度量: 怎么样})}} (y);
  \draw[thick, >=Stealth, ->] (O) to (z) node[anchor = south, align = center] {\texttt{\rmfamily Semantics}\\{\tiny (\texttt{模型: 是什么})}};
}

\coordinate (xz) at (1,0,1);
\coordinate (yz) at (0,1,1);

\uncover<3->{
\draw[canvas is xz plane at y = 0, transform shape, draw = blue, fill = blue!50, opacity = 0.5] (xz) rectangle (O);
\node[canvas is xz plane at y = 0, align = center] at (0.5,0,0.5) {\reflectbox{\texttt{多样化},} \\\reflectbox{\texttt{可调节}}};
}

\uncover<4->{
\draw[canvas is yz plane at x = 0, transform shape, draw = brown, fill = brown!50, opacity = 0.5] (yz) rectangle (O);
\node[canvas is yz plane at x = 0, align = center] at (0,0.5,0.5) {\texttt{精细化},\\\texttt{可度量}};
}
\end{tikzpicture}