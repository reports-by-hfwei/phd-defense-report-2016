\section{本文工作}

%%%%%%%%%%%%%%%%%%%%%%%%%%%%%%%%%%%%%%%%	
\subsection{概述}

%%%%%%%%%%%%%%%
\begin{frame}{工作概述}
  \begin{table}[]
	\centering
	\caption{本文落实\idea{}研究理念.}
	\renewcommand\arraystretch{1.6}
	\resizebox{\textwidth}{!}{%
	  \begin{tabular}{cc|c|c|c|c|}
		\cline{3-6}
		\multicolumn{2}{c|}{} & \multicolumn{2}{c|}{\textbf{读写寄存器}} & \multicolumn{2}{c|}{\textbf{事务}} \\ \cline{3-6} 
		\multicolumn{2}{c|}{} & 多处理器系统 & \cellcolor{red!80}{分布式系统} & \textcolor{gray}{多处理器系统} & \cellcolor{red!80}{分布式系统} \\ \hline

		\multicolumn{2}{|c|}{\textbf{\ideadt{}}} & {\begin{tabular}[c]{@{}c@{}}{\small 相关工作丰富}\\ {\small 理论扎实}\end{tabular}}
		& \begin{tabular}[c]{@{}c@{}}{\small 渐成趋势}\\ {\small 理论欠缺}\end{tabular} 
		& \multirow{2}{*}[-3em]{\begin{tabular}[c]{@{}c@{}}\textcolor{gray}{\small 软件}\\ \textcolor{gray}{\small 事务内存}\end{tabular}}
		& \cellcolor{brown!80}{\begin{tabular}[c]{@{}c@{}}{\small 探索阶段}\\ {(RVSI)}\end{tabular}} \\ \cline{1-4} \cline{6-6}

		\multicolumn{1}{|c|}{\multirow{2}{*}[-1em]{\textbf{\idearm{}}}} & 验证 
		& \begin{tabular}[c]{@{}c@{}}{\small 典型模型}\\ {\small 理论全面}\end{tabular} 
		& \cellcolor{brown!80}{\begin{tabular}[c]{@{}c@{}}{\small 弱模型验证}\\ {(VPC)}\end{tabular}}
		& 
		& \begin{tabular}[c]{@{}c@{}}{\small 理论全面}\\ {\small 指导协议设计}\end{tabular}
		\\ \cline{2-4} \cline{6-6} \hhline{*{3}{~}-*{2}{~}}

		\multicolumn{1}{|c|}{} & 量化 
		& \begin{tabular}[c]{@{}c@{}}{\small 暂无}\\ {\small 强调正确性}\end{tabular}
		& \cellcolor{brown!80}{\begin{tabular}[c]{@{}c@{}}{\small 量化协议}\\ {(PA2AM)}\end{tabular}}
		&  
		& \begin{tabular}[c]{@{}c@{}}{\small 量化协议难}\\ {\small 相关工作少}\end{tabular} 
		\\ \hline
	  \end{tabular}
	}
  \end{table}
\end{frame}
%%%%%%%%%%%%%%%
%%%%%%%%%%%%%%%%%%%%%%%%%%%%%%%%%%%%%%%%
\subsection{VPC: Pipelined-RAM 一致性验证}

\newcommand{\pram}{Pipelined-RAM}
\newcommand{\vpc}[1]{\ifthenelse{\isempty{#1}{}}{\textsf{VPC}}{\textsf{VPC-\MakeUppercase{#1}}}} 
\newcommand{\npc}{$\sf{NP}$-complete}
\newcommand{\npcn}{$\sf{NP}$-completeness}
\newcommand{\rwclosure}{\textsc{RW-Closure}}
\newcommand{\readcentric}{\textsc{Read-Centric}}
%%%%%%%%%%%%%%%
\begin{frame}{VPC 工作在技术框架中的位置}
  \fig{width = 0.50\textwidth}{figures/3d-framework-vpc.pdf}{VPC --- \pram{} 一致性验证.}
\end{frame}
%%%%%%%%%%%%%%%
\begin{frame}{VPC 问题定义}
  \begin{cdef}[VPC: Verifying PRAM Consistency]
    VPC 判定问题:
	\vspace{8pt}
    \begin{description}
	  \setlength{\itemsep}{8pt}
      \item[实例:] 系统执行 {\small (execution $e$; 即, 读写操作序列)}
      \item[问题:] 该执行是否满足 PRAM 一致性模型 {\small ($\mathcal{C}$)}? 
		\[
		  e \in \mathcal{C} \Rightarrow \set{0,1}?
		\]
    \end{description}    
  \end{cdef}

  \vspace{0.30cm}
  实例规模 $n$: 系统执行中操作的总数
\end{frame}
%%%%%%%%%%%%%%%
\begin{frame}{研究动机}
  \question{问题: 为什么要验证 Pipelined-RAM {\small (PRAM)} 一致性?}
  \vspace{0.10cm}

  \begin{description}
    \setlength{\itemsep}{10pt}
    \item[验证:] 用户与商家就数据一致性签订 SLA 协议 \\
	  \citeinbeamer{Amazon}{SOSP}{07} \citeinbeamer{Golab}{PODC}{11}
	  \vspace{2pt}
	  \begin{itemize}
		\item \textcolor{teal}{[商家]} 系统测试手段之一
		\item \textcolor{teal}{[用户]} 确认系统是否提供了其所声称的数据一致性 
	  \end{itemize}
	\pause
	\item[PRAM:] 存储系统常提供``会话'' {\small (session)} 一致性\\
      \citeinbeamer{Saito}{CSUR}{05} \citeinbeamer{Terry}{CACM}{13}
	  \vspace{2pt}
      \begin{itemize}
		\item 包含了弱一致性的诸多变体 
		\item 近似于 PRAM 一致性 \citeinbeamer{Brzezi$\acute{\text{n}}$ski}{PDP}{04} \citeinbeamer{Bailis}{VLDB}{13}
      \end{itemize}
  \end{description}
\end{frame}
%%%%%%%%%%%%%%%
\begin{frame}{VPC 问题分类}
  \begin{table}[!t]
    \centering
	\caption{VPC 问题的四种变体 (按``执行''的类型) 及复杂度分析 ($\textcolor{red}{[\ast]}$\textcolor{red}{: 本文工作}).}
	\renewcommand\arraystretch{1.2}
    \begin{tabular}{|c|c|c|}
      \hline
      & \it (S)ingle variable  & \it (M)ultiple variables
      \\ \hline
	  {\it write (D)uplicate values} &
	  \innercell{c}{VPC-SD \\ (\npc{}) $\textcolor{red}{[\ast]}$} &
	  \innercell{c}{VPC-MD \\ (\npc{}) $\textcolor{red}{[\ast]}$}
      \\ \hline
	  \only<1>{\it write (U)nique value}\only<2>{\cellcolor{brown!80}{\it write (U)nique value}} &
      \innercell{c}{VPC-SU \\ (P) \citeinbeamer{Golab}{PODC}{11}} &
      \innercell{c}{VPC-MU \\ (P) $\textcolor{red}{[\ast]}$}
      \\ \hline
    \end{tabular}
  \end{table}

  \vspace{10pt}
  
  \uncover<2->{\textcolor{brown}{\centerline{Read-mapping \citeinbeamer{Gibbons}{SICOMP}{97}: $\forall r, f(r) = w$.}}}
\end{frame}
%%%%%%%%%%%%%%%
\begin{frame}{VPC-SD (VPC-MD) 是 \npc{} 问题}
  \fig{width = 0.40\textwidth}{figures/vpcsd-npc.pdf}{对应于 \textsc{Unary 3-Partition} 实例 $A = \{2,2,1,1,1,1\}, m = 2, B = 4$ 的 VPC 执行.} 
\end{frame}
%%%%%%%%%%%%%%%
\begin{frame}{VPC-MU 的多项式算法 \rwclosure{}}
  \begin{figure}[h!]
    \centering
    \begin{adjustbox}{max totalsize = {0.65\textwidth}{1.00\textheight}, center}
	  \input{tikz-in-beamer/rw-closure-example}
    \end{adjustbox}
	\caption{\rwclosure{} 算法示例: 在传递闭包之上迭代应用 $w'wr$ 规则.}
  \end{figure}

  TODO: 规则c
\end{frame}
%%%%%%%%%%%%%%%
\begin{frame}{VPC-MU 的多项式算法 \rwclosure{}}
  \begin{ctheorem}[\rwclosure{} 算法正确性]
	\vpc{mu} 实例满足 \pram{} 一致性当且仅当 \rwclosure{} 算法所得图是有向无环图.
  \end{ctheorem}

  \pause 
  TODO: proof

  \pause
  \centerline{\rwclosure{} 算法复杂度: $O(n^2) \cdot O(n^3) = O(n^5)$}
\end{frame}
%%%%%%%%%%%%%%%
\begin{frame}{VPC-MU 的多项式算法 \readcentric{}}
  关键点:

  \begin{ctheorem}[\rwclosure{} 算法正确性]
	\vpc{mu} 实例满足 \pram{} 一致性当且仅当 \readcentric{} 算法所得图是有向无环图.
  \end{ctheorem}

  \pause
  \centerline{\rwclosure{} 算法复杂度: }
\end{frame}
%%%%%%%%%%%%%%%
\begin{frame}{实验评估}
  
\end{frame}
%%%%%%%%%%%%%%%
\begin{frame}{VPC 的意义}
  \mdf{red}{blue}{}{teal}{
	\begin{enumerate}
	  \item \readcentric{} 算法可用于测试系统是否正确实现了 \pram{} 一致性模型
	  \item \npcn{} 结果有助于理解弱一致性模型的复杂度
	\end{enumerate}
  }
\end{frame}
%%%%%%%%%%%%%%%

%%%%%%%%%%%%%%%%%%%%%%%%%%%%%%%%%%%%%%%%	
\subsection{PA2AM: (2-)Atomicity 一致性维护与量化}

%%%%%%%%%%%%%%%
\begin{frame}{PA2AM 工作在技术框架中的位置}
  \fig{width = 0.50\textwidth}{figures/3d-framework-pa2am.pdf}{PA2AM --- (2-)Atomicity 一致性维护与量化.}
\end{frame}
%%%%%%%%%%%%%%%
\begin{frame}{研究动机}
  \question{问题: 为什么提出 probabilistically-atomic 2-atomicity {\small (PA2AM)} 一致性?}
  \vspace{0.30cm}

  \pause
  ``数据一致性/访问延迟'' PACELC 权衡 \citeinbeamer{Abadi}{IEEE Computer}{12}:

  \fignocaption{width = 0.35\textwidth}{figures/stronger-consistency-tradeoff.pdf}

  % \begin{quote}
  %   ``As soon as a distributed storage system replicates data, a \textcolor{brown}{tradeoff 
  %   between consistency and latency} arises.''
  % \end{quote}

  \pause

  ``低延迟''至关重要:
  \begin{itemize}
	\item 100ms 额外延迟 $\Rightarrow$ 1\% 销售下滑 \citeinbeamer{Amazon}{Blog}{06}
	\item 100$\sim$400ms 额外延迟 $\Rightarrow$ 0.2\%$\sim$0.6\% 搜索量下降 \citeinbeamer{Google}{Blog}{09}
  \end{itemize}

  % {\small
  % \begin{table}
  %   \begin{tabular}{c|c}
  %     \textcolor{blue}{\bf 系统} & \textcolor{blue}{\bf 一致性}		\\ \hline
  %     Dynamo@Amazon \citeinbeamer{Amazon}{SOSP}{07} & eventual consistency \\ \hline
  %     PNUTS@Yahoo! \citeinbeamer{Yahoo!}{PVLDB}{08} & cache consistency \\ \hline
  %     Tao@Facebook \citeinbeamer{Facebook}{ATC}{13} & $\le$ read-after-write \\ \hline
  %   \end{tabular}
  % \end{table}
  % }
\end{frame}
%%%%%%%%%%%%%%%
\begin{frame}{PA2AM 一致性}
  % \begin{cdef}[``近乎强''一致性]
  %   对某特定强一致性的弱化: \begin{itemize}
  %     \item (版本) 允许读陈旧值,但陈旧度有限
  %     \item (概率) 读到陈旧值的概率很小
  %   \end{itemize}
  % \end{cdef}

  \begin{center}
	\only<2->{\textcolor{blue}{``近乎强''一致性: }}{在保证\textcolor{red}{低延迟}的情况下获得\textcolor{red}{尽可能强}的数据一致性.}
  \end{center}

  \uncover<3->{
  \begin{cdef}[PA2AM 一致性]
	\begin{description}
	  \setlength{\itemsep}{5pt}
	  \item[低延迟:] \uncover<4->{读操作只需一轮网络通信} 
	  \item[尽可能强:] \uncover<6->{对 atomicity {\small (strongest)} 的弱化}
    \begin{itemize}
	  \item<6-> \textcolor{brown}{\it (版本)} 2-atomicity: 允许读陈旧值, 但\textcolor{red}{陈旧度 $k \le 2$}
	  \item<6-> \textcolor{brown}{\it (概率)} \textcolor{red}{$\mathbb{P}(k = 2)$ 很小}
    \end{itemize}
    \end{description}
  \end{cdef}
  }

  \vspace{0.20cm}
  \uncover<5->{
  \begin{ctheorem}[不可能性结果]
    (单写模型下) 不存在低延迟的 atomicity 维护算法 \citeinbeamer{Dutta}{PODC}{04}.
  \end{ctheorem}
  }
\end{frame}
%%%%%%%%%%%%%%%
\begin{frame}{PA2AM 维护算法}
  \fig{width = 0.80\textwidth}{figures/atomicity-2am-read-compare.pdf}
  {经典 atomicity 算法中, 读操作需两轮网络通信 \citeinbeamer{Attiya}{JACM}{95} 
  \citeinbeamer{Dutta}{PODC}{04}. 
  PA2AM 算法实现 2-atomicity {\scriptsize (单写模型下)}, 读操作只需一轮网络通信.}
\end{frame}
%%%%%%%%%%%%%%%
\begin{frame}{PA2AM 量化分析}
  \question{问题: PA2AM 算法在多大程度上违反了 atomicity?}
  \vspace{0.10cm}

  \begin{itemize}
    \setlength{\itemsep}{10pt}
	\item 充要条件: ONI (old-new inversion) \citeinbeamer{Attiya}{JACM}{95}
      \fignocaption{width = 0.45\textwidth}{figures/2atomicity-case.pdf}
    \item PA2AM 量化分析: 计算 $\mathbb{P}(\textrm{ONI})$, 其值越小越好
      \begin{enumerate}
        \setlength{\itemsep}{3pt}
        \item $\textrm{ONI} \triangleq \textrm{CP} \cap \textrm{RWP}$
        \item 排队论建模, 计算 $\mathbb{P}(\textrm{CP})$ \item 带时间的球盒模型, 计算 
          $\mathbb{P}(\textrm{RWP|CP})$
        % \item 实验统计 $\mathbb{P}(\textrm{CP})$, $\mathbb{P}(\textrm{RWP|CP})$ 与 
        % $\mathbb{P}(\textrm{ONI})$, 以作对照
      \end{enumerate}
  \end{itemize}
\end{frame}
%%%%%%%%%%%%%%%
\begin{frame}{PA2AM 量化分析}
  公式推导:
  \begin{figure}
	\begin{subfigure}{0.50\textwidth}
	  \centering
	  \includegraphics[width = 0.70\textwidth]{figures/cp.png}
	\end{subfigure}%
	\begin{subfigure}{0.50\textwidth}
	  \centering
	  \includegraphics[width = 0.70\textwidth]{figures/rwp.png}
	\end{subfigure}
  \end{figure}

  \vspace{0.10cm}

  数值结果 (左一) 与实验结果 (右二): 
  \begin{figure}
	\begin{subfigure}{0.45\textwidth}
	  \centering
	  \includegraphics[width = 0.60\textwidth]{figures/oni-pgfplot.pdf}
	\end{subfigure}%
	\begin{subfigure}{0.55\textwidth}
	  \centering
	  \includegraphics[width = 0.90\textwidth]{figures/experiment-oni-pgfplot.pdf}
	\end{subfigure}
  \end{figure}
\end{frame}
%%%%%%%%%%%%%%%
\begin{frame}{PA2AM vs. 弱一致性模型}
  \begin{figure}
	\begin{subfigure}{0.50\textwidth}
	  \centering
	  \includegraphics[width = 0.85\textwidth]{figures/rwn-2am-read-latency-quantiles.pdf}
	  \caption{读操作延迟对比.}
	\end{subfigure}%
	\begin{subfigure}{0.50\textwidth}
	  \centering
	  \includegraphics[width = 0.70\textwidth]{figures/rwn-maj.pdf}
	  \caption{$k$-陈旧度对比.}
	\end{subfigure}
	\caption{PA2AM 与 eventual consistency (RWN-Maj 协议) 对比结果.}
  \end{figure}

  \vspace{-0.50cm}
  \begin{table}[]
  \renewcommand{\arraystretch}{1.3}
  \centering
  \caption{不同 $R,W,N$ 配置下, RWN-All 执行中具有不同陈旧度的读操作的比率.}
  \label{tbl:rwn-all-staleness}
  \resizebox{\textwidth}{!}{%
  \begin{tabular}{|c||c|c|c|c||c|c|c|c|c|c|c|}
  \hline
  {\bfseries \# replicas} & \multicolumn{4}{c||}{\bfseries replica factor = 3 ($400,000$
    \textsl{read} operations)} & \multicolumn{7}{c|}{\bfseries replica factor = 5 ($800,000$
    \textsl{read} operations)} 
  \\ \hline
  {\bfseries $R,W,N$} % \textrm{R}+\textrm{W} $\boldsymbol{\le}$ \textrm{N}}  
  & $1+1<3$	& $1+2=3$	& $2+1=3$	& $2+2>3 \;(\text{PA2AM})$ 
  & $1+2<5$ 	& $1+3<5$ 	& $1+4=5$	& $2+2<5$ 	& $2+3=5$    & $3+2=5$     & $3+3>5 \;(\text{PA2AM})$     
  \\ \hline \hline
  {\boldmath $\max k$}          & $6$      & $4$ 	& $2$    &  {\boldmath $1$}
  & $2$	& $2$      	& $2$	   & $2$	& $2$	 &  $2$     & {\boldmath $1$}
  \\ \hline
  {\bfseries $\sum_{k \ge 1}$-staleness}       & $0.0084125$      & $0.000315$ 	&  $0.0004675$	&
  {\boldmath $0.000085$}
  	&  $0.00377875$	&  $0.002755$	&  $0.00406$     & $0.0027225$      & $0.0020275$     & $0.002255$      
	&  {\boldmath $0.0003525$}
  \\ \hline
  \end{tabular}%
}
\end{table}

\end{frame}
%%%%%%%%%%%%%%%
\begin{frame}{PA2AM 的意义}
  \mdf{red}{blue}{PA2AM 可作为一致性/延迟权衡的一种有价值的选项}{teal}{
	\begin{itemize}
	  \item 既 {\small (在统计意义上)} 满足强一致性模型对数据一致性的高标准
	  \item 又具有弱一致性模型的性能优势
    \end{itemize}
  }
\end{frame}
%%%%%%%%%%%%%%%

%%%%%%%%%%%%%%%%%%%%%%%%%%%%%%%%%%%%%%%%	
\subsection{RVSI: Snapshot Isolation 一致性弱化与维护}

\newcommand{\chameleon}{$\textsc{Chameleon}^{\textsc{\scriptsize TKVS}}$}
\newcommand{\konebv}{$k_1$-BV}
\newcommand{\ktwofv}{$k_2$-FV}
\newcommand{\kthreesv}{$k_3$-SV}
\newcommand{\mpord}[1]{\mathcal{O}_{#1}}
\newcommand{\tsts}[1]{#1.{sts}}
\newcommand{\tcts}[1]{#1.{cts}}
%%%%%%%%%%%%%%%
\begin{frame}{RVSI 工作在技术框架中的位置}
  \fig{width = 0.50\textwidth}{figures/3d-framework-rvsi.pdf}
	{RVSI --- Snapshot Isolation 一致性弱化与维护.}
\end{frame}
%%%%%%%%%%%%%%%
\begin{frame}{研究动机}
  \question{问题: 为什么要提出 Relaxed Version Snapshot Isolation {\small (RVSI)} 一致性?}
  \vspace{0.15cm}

  \begin{description}
    \setlength{\itemsep}{5pt}
    \item[分布式事务:]
      \begin{itemize}
        \item ``all-or-none'' 语义
		\item 受到分布式存储系统的关注 \href{https://issues.apache.org/jira/browse/CASSANDRA-7056}{\citeinbeamer{Cassandra}{CASSANDRA-ISSUE-7056}{14}}
      \end{itemize}
	\item[弱一致性:] \textcolor{red}{SI} \citeinbeamer{Berenson}{SIGMOD}{95} \textcolor{red}{$\xRightarrow{\text{relaxed}}$}\\
        PCSI \citeinbeamer{Elnikety}{SRDS}{05} PSI \citeinbeamer{Sovran}{SOSP}{11} NMSI \citeinbeamer{Ardekani}{SRDS}{13} 
    \pause
	\vspace{0.30cm}
	\item[\textcolor{red}{异常控制:}] 容忍``有限度的''异常 \citeinbeamer{Yu}{TOCS}{02}
	\item[\textcolor{red}{可调节:}] 
      \begin{itemize}
        \item 不同应用对一致性需求不同 \citeinbeamer{Terry}{CACM}{13}
        \item 运行时决定 \citeinbeamer{Terry}{SOSP\&TR}{13}
      \end{itemize}
  \end{description}
\end{frame}
%%%%%%%%%%%%%%%
\begin{frame}{RVSI 定义}
  RVSI {\small (Relaxed Version Snapshot Isolation)} 定义原则:
  \begin{itemize}
    \item<1-> 参数 $k_1, k_2, k_3$ 调节/控制相对于 SI 的``异常''程度
    \item<2-> $\text{RC} \supset \text{RVSI}(k_1, k_2, k_3) \supset \text{SI}$
    \item<2-> $\text{RVSI}(\infty,\infty,\infty) = \text{RC}; \qquad \text{RVSI}(1,0,\ast) = \text{SI}$
  \end{itemize}

  % \vspace{0.60cm}

  % \uncover<1->{
  % \begin{description}
  %   \item[RC {\small (Read Committed)}:] 读取成功提交的更新
  %   \item[SI {\small (Snapshot Isolation)}:] 观察到事务开始之前的最新的系统快照 + 无写冲突事务
  % \end{description}
  % }
\end{frame}
%%%%%%%%%%%%%%%
\begin{frame}{RVSI 定义}
  SI: ``观察到事务开始 \textcolor{blue}{\scriptsize (2) }\textcolor{red}{之前的} 
	\textcolor{blue}{\scriptsize (1) }\textcolor{red}{最新的}系统 \textcolor{blue}{\scriptsize (3) }\textcolor{red}{快照}''
  \begin{cdef}[RVSI {\small (弱化 SI 的要点)}]
	\vspace{5pt}

    \begin{description}
      \item[单变量读 $\texttt{read}(x)$:] \hfill 
        \begin{enumerate}
		  \item 允许读 $\le k_1$ 陈旧值 (\konebv{})
		  \item 允许读 $\le k_2$ 并发更新 (\ktwofv{})
        \end{enumerate}
      \item[多变量读 $\texttt{read}(x), \texttt{read}{(y)}$:] \hfill
        \begin{enumerate}
          \setcounter{enumi}{2}
		\item $\textsl{dist}(\textsl{snap}{(x)},\textsl{snap}{(y)}) \le k_3$ (\kthreesv{})
        \end{enumerate}
    \end{description}
  \end{cdef}
\end{frame}
%%%%%%%%%%%%%%%
\begin{frame}{\chameleon{}~\footnotemark[1]分布式事务键值存储原型系统设计}
  \begin{description}
	\item[系统架构:] 阿里云~\footnotemark[2]多数据中心 {\small ($9 = 3 \times 3$)} % ~\footnotemark[2]\footnotetext[2]{阿里云: \url{https://www.aliyun.com/}.} 
	\item<2->[数据分区:] 同一数据中心
	\item<2->[数据副本:] 跨数据中心; 主从结构
  \end{description}

  \footnotetext[1]{\url{https://github.com/hengxin/chameleon-transactional-kvstore}}
  \footnotetext[2]{\url{https://www.aliyun.com/}}

  \fignocaption{width = 0.55\textwidth}{figures/chameleon-arch.pdf}
\end{frame}
%%%%%%%%%%%%%%%
\begin{frame}{RVSI 维护算法}
  \[
    \textcolor{blue}{\text{RC} \supset \text{RVSI}(k_1, k_2, k_3) \supset \text{SI}}
  \]

  \vspace{0.10cm}

  RVSI 维护算法:
  \begin{itemize}
    \item 以分布式 RC 和 SI 协议 为基础
	  \pause
    \item 事务提交前, 计算 RVSI ``版本约束'' ($k_1, k_2, k_3$ 相关不等式)
	  \begin{description}
		\item[\konebv{}:] $\mpord{x}(\tsts{T_i}) - \mpord{x}(\tcts{T_j}) < k_1$
		\item[\ktwofv{}:] $\mpord{x}(\tcts{T_j}) - \mpord{x}(\tsts{T_i}) \le k_2$
		\item[\kthreesv{}:] $\mpord{x}(\tcts{T_l}) - \mpord{x}(\tcts{T_j}) \le k_3$
	  \end{description}
    \item 事务提交时, 检查 RVSI ``版本约束''
  \end{itemize}

  % \fignocaption{width = 0.50\textwidth}{figures/chameleon-build-passing.png}
  % \textcolor{red}{\small \url{https://github.com/hengxin/chameleon-transactional-kvstore}}
\end{frame}
%%%%%%%%%%%%%%%
\begin{frame}{RVSI 维护算法}
  \begin{description}
	% \item[系统组件:] 客户端库 + 数据中心
	\item[数据分区:] 分布式事务原子提交协议 {\small (2PC)}
	\item[数据副本:] 懒惰复制 {\small (lazy replication)} 协议
  \end{description}

  \fignocaption{width = 0.42\textwidth}{figures/chameleon-framework.pdf}
\end{frame}
%%%%%%%%%%%%%%%
\begin{frame}{RVSI 实验评估}
  \begin{table}[]
  \renewcommand{\arraystretch}{1.1}
  \centering
  \caption{事务负载参数表.\protected\\(\textcolor{blue}
	{评估目标: RVSI 对事务中止率的影响})}
  \resizebox{\textwidth}{!}{%
  \begin{tabular}{|c||c|c|c|}
	\hline
	{\bfseries Parameter}   & {\bfseries F(ixed)/V(ariable)/R(andom)}	
	& {\bfseries Value}		& {\bfseries Explanation}
	\\ \hline  \hline
	\#keys  				& F		& 5  				&  	size of keyspace
	\\ \hline
	\cellcolor{brown}mpl	& \textcolor{red}{V}		& 5, 10, 15, 20, 25, 30
	& \innercell{c}{multiprogramming level: \\ number of concurrent clients} \\ \hline
	\#txs/client					& F		& 1000 						
	& \innercell{c}{number of txs per client}
	\\ \hline
	\#ops/tx					& R		& $\sim$ Binomial(20, 0.5)	
	&  \innercell{c}{number of operations per tx}
	\\ \hline
	\cellcolor{brown}rwRatio & \textcolor{red}{V} 
	  & 1:2, 1:1, 4:1 & {\#reads}/{\#writes}
	\\ \hline
	zipfExponent			& F		& 1		& parameter for Zipfian distribution
	\\ \hline  \hline
	\cellcolor{brown}$(k_1, k_2, k_3)$		& \textcolor{red}{V}
		&  \innercell{c}{(1,0,0) \\ (1,0,2) (1,1,0) \\ (2,0,0) (2,1,2) (2,2,1)}	
		&  for \konebv{}, \ktwofv{}, and \kthreesv{}
	\\ \hline  \hline
	minInterval				& F		& 0ms		& minimum inter-transactions time
	\\ \hline
	maxInterval				& F		& 10ms		& maximum inter-transactions time
	\\ \hline
	meanInterval			& R		& 5ms		
	& \innercell{c}{mean inter-transactions time \\ for exponential distribution}
	\\ \hline
  \end{tabular}
  }
\end{table}

\end{frame}
%%%%%%%%%%%%%%%
\begin{frame}{RVSI 实验评估}
  \fig{width = 0.70\textwidth}{figures/rvsi-rw4-abort-rates.pdf}
  {\textcolor{teal}{读频繁} (rwRatio = 4:1) 负载下 RVSI 对事务中止率的影响.}

	\pause
  \begin{description}
	\item[\textcolor{blue}{wcf-aborted:}] 无显著变化 {\small ($\text{mpl} = 30\text{ 时}, \text{wcf}(1,0,0) = 0.184733$)}
	  \pause
	\item[\textcolor{red}{vc-aborted:}] 显著减少 {\small ($\text{mpl} = 30 \text{ 时}, 
	  \text{vc}(1,0,0) = 0.204733; \text{vc}(1,1,0) = 0.066433; \text{vc}(2,2,1) = 0.002033$)}
  \end{description}
\end{frame}
%%%%%%%%%%%%%%%
\begin{frame}{RVSI 实验评估}
  \begin{columns}
	\column{0.48\textwidth}
	  \fig{width = 1.00\textwidth}{figures/rvsi-rw05-abort-rates.pdf}
		{写频繁 (rwRatio = 1:2) 负载下 RVSI 对事务中止率的影响.}
	\column{0.48\textwidth}
	  \fig{width = 1.00\textwidth}{figures/rvsi-rw1-abort-rates.pdf}
		{读写相当 (rwRatio = 1:1) 负载下 RVSI 对事务中止率的影响.}
  \end{columns}

  \begin{description}
	\item[\textcolor{blue}{wcf-aborted:}] 无显著变化
	\item[\textcolor{red}{vc-aborted:}] 绝对数值小; 相对变化显著 
  \end{description}
\end{frame}
%%%%%%%%%%%%%%%
\begin{frame}{RVSI 的意义}
  \mdf{red}{blue}{RVSI 对事务中止率的影响}{teal}{
	\begin{enumerate}
	  \item 适当放松事务对 RVSI 版本规约的要求可降低事务中止率
	  \item RVSI 能否``显著''降低事务中止率与负载类型相关
    \end{enumerate}
  }
\end{frame}
%%%%%%%%%%%%%%%

%%%%%%%%%%%%%%%
