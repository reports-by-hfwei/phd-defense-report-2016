\section{研究方法}

%%%%%%%%%%%%%%%%%%%%%%%%%%%%%%
\subsection{理论模型:分布共享数据}

%%%%%%%%%%%%%%%
% \begin{frame}{理论模型}
%   分布数据一致性:
%   \begin{enumerate}
% 	\item 精细化, 可度量
% 	\item 多样化, 可调节
%   \end{enumerate}
% \end{frame}
%%%%%%%%%%%%%%%
\begin{frame}{分布共享数据模型}
  \begin{columns}
	\column{0.50\textwidth}
	  \fignocaption{width = 0.60\textwidth}{figures/shared-data-clients.pdf}
	  \begin{center}
		共享数据系统 (single copy)
	  \end{center}
	\pause
	\column{0.50\textwidth}
	  \fignocaption{width = 0.70\textwidth}{figures/distributed-data-clients.pdf}
	  \begin{center}
		分布数据系统 (replicas)
	  \end{center}
  \end{columns}
\end{frame}
%%%%%%%%%%%%%%%
\begin{frame}{分布共享数据模型}
  \fignocaption{width = 0.48\textwidth}{figures/distributed-shared-data-clients.pdf}
  \begin{center}
	\textcolor{blue!80}{分布共享数据模型:} 在分布数据之上提供共享数据的假象
  \end{center}
\end{frame}
%%%%%%%%%%%%%%%
\begin{frame}{分布共享数据模型}
  \begin{center}
    \textcolor{cyan}{$p_0, p_1:$} 客户进程 \hspace{0.30cm} \textcolor{cyan}{$x, y:$} 共享变量 
  \end{center}
  \pause

  多进程并发读/写共享变量:
  
  \fignocaption{width = 0.55\textwidth}{figures/register-what-value.pdf}
  \pause

  \begin{center}
    \question{数据一致性问题: 读操作允许返回什么值?}
  \end{center}

  \answer{
  \[
    \text{不同一致性模型}
    \xrightleftharpoons[\text{定义}]{\,\text{规定}\,}
    \text{不同合法返回值}
  \]} 
\end{frame}
%%%%%%%%%%%%%%%
\begin{frame}{分布共享数据服务}
  \fignocaption{width = 0.40\textwidth}{figures/dsds.pdf}
\end{frame}
%%%%%%%%%%%%%%%
\begin{frame}{分布共享数据服务}
  \begin{columns}
	\column{0.60\textwidth}
	\begin{center}
	  \textcolor{red}{\large 分布共享数据服务:}\\
	  \vspace{0.50cm} 
	  分布共享内存模型 (多处理器系统)\\
	  \textcolor{blue}{[传统概念]}\\
	  \vspace{0.20cm} +\\ \vspace{0.20cm}
	  分布数据系统\\
	  \textcolor{blue}{[新平台]}
	\end{center}
	\column{0.40\textwidth}
	  \fignocaption{scale = 0.30}{figures/more-old-wine-new-bottle-poster.jpg}
  \end{columns}
\end{frame}
%%%%%%%%%%%%%%%
\begin{frame}{我们的工作}
  \begin{columns}
	\column{0.40\textwidth}
	  \fignocaption{scale = 0.30}{figures/more-old-wine-new-bottle-poster.jpg}
	\column{0.40\textwidth}
	  实现分布共享数据服务:
	  \begin{enumerate}
		\item 精细化, 可度量
		\item 多样化, 可调节
	  \end{enumerate}
  \end{columns}
\end{frame}
%%%%%%%%%%%%%%%
%%%%%%%%%%%%%%%%%%%%%%%%%%%%%%
\subsection{技术途径: 三维框架}

%%%%%%%%%%%%%%%
\begin{frame}{分布共享内存中的数据一致性问题}
      数据一致性问题的三个层面: 
      \begin{description}
	\item[1. 虚拟共享数据有什么?]
	  \begin{itemize}
	    \item 数据类型
	  \end{itemize}
	\item[2. 上层接口语义是什么?]
	  \begin{itemize}
	    \item 一致性模型
	  \end{itemize}
	\item[3. 底层消息传递为什么?]
	  \begin{itemize}
	    \item 一致性保障
	  \end{itemize}
      \end{description}
\end{frame}
%%%%%%%%%%%%%%%
\begin{frame}{研究框架}
  \fig{width = 0.75\textwidth}{figures/thesis-proposal-3d-framework-allinone.pdf}{数据一致性及保障技术研究框架}
\end{frame}
%%%%%%%%%%%%%%%
