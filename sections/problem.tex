\section{研究问题}

%%%%%%%%%%%%%%%
\begin{frame}{面向共享数据的编程模型}
  \begin{columns}
    \column{0.50\textwidth}
      \fig{width = 0.60\textwidth}{figures/shared-data.png}{(集中式)共享数据系统 \todo{重绘}.}
    \column{0.50\textwidth}
      \textcolor{red}{共享变量}编程模型 (\todo{happy programmer}): 
      \begin{itemize}
	\item 数据层: 数据建模为变量
	\item 业务层: 读---计算---写
	\item 变量全局唯一 (one-copy)
	\item \textcolor{cyan}{\footnotesize 隐含假设:} 变量随时可访问
	\item 读操作返回最新值 
      \end{itemize}
  \end{columns}
\end{frame}
%%%%%%%%%%%%%%%
\begin{frame}{面向分布数据的编程模型 (I)}
  \begin{columns}
    \column{0.50\textwidth}
      \fig{width = 0.80\textwidth}{figures/distributed-data.pdf}{分布数据系统.}
    \column{0.50\textwidth}
      共享变量编程模型与分布数据不匹配:
      \begin{itemize}
	\item 数据副本 (replication)
	\item 没有全局唯一变量的概念
	\item 节点/通讯故障
	\item 读操作语义无定义
      \end{itemize}
  \end{columns}
\end{frame}
%%%%%%%%%%%%%%%
\begin{frame}{面向分布数据的编程模型 (II)}
  \begin{columns}[t]
    \column{0.50\textwidth}
      \textcolor{red}{消息传递}编程模型:
      \begin{itemize}
	\item 读/写 + 通信 (communication)
	\item 从哪里读
	\item 写哪里去
	\item 如何理解返回值
	\item 如何处理失败
      \end{itemize}
    \column{0.50\textwidth}
      \pause
      消息传递编程模型的缺点: (\todo{unhappy programmer})
      \begin{description}
	\item[数据层:] 暴露分布数据细节
	\item[业务层:] 难于编程
	\item[数据层:] 数据语义不明确
	\item[业务层:] 难于保证程序正确性
      \end{description}
  \end{columns}
\end{frame}
%%%%%%%%%%%%%%%
\begin{frame}{面向分布数据的编程模型 (III)}
  \begin{center}
    共享变量编程模型与分布数据不匹配的根本原因: 

    \textcolor{blue}{\large 分布数据一致性问题}
  \end{center}

  \begin{cdef}[数据一致性问题 (非形式化定义)]
    读操作语义问题: 在分布数据环境下, 读操作允许返回什么值?
  \end{cdef}

  \pause
  \begin{alertblock}{如何理解数据一致性问题:}
    \begin{enumerate}
      \item 数据层数据一致与否是相对于业务层逻辑而言的
      \item 可通过业务层出现的异常行为(anomalies)来衡量数据是否一致
    \end{enumerate}
  \end{alertblock}
\end{frame}
%%%%%%%%%%%%%%%
\begin{frame}{数据一致性问题举例 (I)}
  \fig{width = 0.60\textwidth}{figures/data-inconsistency-comment-reordering.pdf}
  {社交网络中, 消息-评论乱序 \citeinbeamer{Lloyd}{CACM}{14}.}
\end{frame}
%%%%%%%%%%%%%%%
\begin{frame}{数据一致性问题举例 (II)}
  \begin{figure}[h!]
    \centering
    \begin{adjustbox}{max totalsize = {0.65\textwidth}{0.65\textheight}, center}
      %        File: data-inconsistency-rmw.tex
%     Created: Thu Oct 08 10:00 AM 2015 C
% Last Change: Thu Oct 08 10:00 AM 2015 C
%
\begin{tikzpicture}
  \uncover<1->{
  \begin{scope}
    % phone
    \node (phone) [] at (0,0) {\includegraphics[scale = 0.4]{figures/phone-icon.png}};
    % tablet
    \node (tablet) [below left = 4.0cm and 3.0cm of phone] {\includegraphics[scale = 
    0.55]{figures/tablet-icon.jpg}};
    % pc
    \node (pc) [below right = 4.0cm and 3.0cm of phone] {\includegraphics[scale = 
    0.60]{figures/pc-icon}};
  \end{scope}
  }

    % write from pc 
  \uncover<2->{
    \begin{scope}[<->, blue, line width = 6pt, font = \huge]
      % person at pc
      \only<2>{
      \node (person-pc) [above left = -1.0cm and -2.0cm of pc] {\includegraphics[scale = 
      0.30]{figures/person-icon}};
      }
      \draw [] (pc.west) to node [midway, sloped, above] {\textbf{1.} update file $f$} 
      (tablet.east);
      \draw [red, loosely dashed] (pc.north) to [out = 90, in = -20] node [midway, sloped, font = 
      \Huge, scale = 2]{$\times$} (phone.east);
      \draw (pc) to [loop, out = 60, in = -10, looseness = 3] node [midway, above, sloped] 
      {\textbf{1.} update file $f$} (pc);
    \end{scope}
   }


    % read from phone
   \uncover<3->{
    \begin{scope}[<->, brown, line width = 6pt, font = \huge]

      % person at phone
      \node (person-phone) [above left = -1.0cm and -2.0cm of phone] {\includegraphics[scale = 
      0.30]{figures/person-icon}};

      \draw (phone) to [loop, out = 180, in = -100, looseness = 3] node [midway, below, sloped] 
      {\textbf{2.} read file $f$} (phone);   
    \end{scope}

    % update lost
    \node (inconsistency) [font = \huge, right = of person-phone, red] {\bf Update Lost!};
  }
\end{tikzpicture}

    \end{adjustbox}
    \caption{多设备文件共享时, 更新丢失 ($\#N = 3, \#W = 2, \#R = 1$).}
  \end{figure}
\end{frame}
%%%%%%%%%%%%%%%
\begin{frame}{面向分布数据的编程模型 (IV)}
  \begin{center}
    共享变量编程模型与分布数据不匹配: 数据一致性问题 
  \end{center}

  \vspace{0.50cm}
  消息传递编程模型: 业务层显式处理数据一致性问题
  \begin{itemize}
    \item 理论难度大, 实现繁琐
    \item 处理不当: 异常
  \end{itemize}
\end{frame}
%%%%%%%%%%%%%%%
\begin{frame}{面向分布数据的编程模型 (V)}
  \begin{cquestion}
    如何解决共享变量编程模型与分布数据不匹配的问题?
  \end{cquestion}

  \vspace{0.30cm}
  \begin{center}
    {\large \textcolor{red}{分布共享数据}服务: 在\textcolor{blue}{分布数据}之上提供\textcolor{blue}{共享数据}的假象}
  \end{center}
  
  \begin{description}
    \item<2->[业务层:] 采用``自然的''共享变量编程模型
    \item<3->[分布共享数据服务:] 专注分布数据一致性问题, 实现分布数据透明性
      \begin{itemize}
	\item 屏蔽消息传递编程模型的通信细节
	\item 形式化定义读操作语义
      \end{itemize}
    \item<2->[数据层:] 分布数据
  \end{description}
\end{frame}
%%%%%%%%%%%%%%%
\begin{frame}[fragile]{分布共享数据服务应用举例 (I)}
   \lstset{language=C++,
           basicstyle=\ttfamily\scriptsize,
           keywordstyle=\color{blue}\ttfamily,
           stringstyle=\color{red}\ttfamily,
           commentstyle=\color{green}\ttfamily,
	   breaklines=true,
	   captionpos=b,
	   caption=在社交网络应用中使用满足``因果一致性''的共享变量.
          }
    \begin{lstlisting}
      // Alice's post
      put(key = lost_msg, val = "lost")
      put_after(key = found_msg, val = "found", dep = lost_msg)

      // Bob's reply
      get(key = found_msg)  // get "found"
      put(key = glad_msg, val = "glad", dep = found_msg)

      // Eve's view
      read(key = glad_msg)  // get "glad"
      read(key = found_msg) // get "found" instead of "NULL"
    \end{lstlisting}
\end{frame}
%%%%%%%%%%%%%%%
\begin{frame}{分布共享数据服务应用举例 (II)}
  Another example: coordination???
\end{frame}
%%%%%%%%%%%%%%%
