\section{研究问题}

%%%%%%%%%%%%%%%
\begin{frame}{数据一致性问题}
  \todo{图: 从分布到共享} 

  \begin{center}
	读操作语义问题: 在分布数据环境下, 读操作允许返回什么值?
  \end{center}

  \pause
  \vspace{1.00cm}

  \begin{center}
	\textcolor{blue}{\Large 数据一致性问题}
  \end{center}
\end{frame}
%%%%%%%%%%%%%%%
\begin{frame}{数据一致性问题}
  数据一致与否是相对于应用逻辑而言的:
  \pause
  \begin{itemize}[<+->]
	\setlength\itemsep{8pt}
	\item 数据一致性模型多样
	\item 数据一致性模型有强弱之分
	\item 应用规约数据一致性需求
	\item 数据不一致导致应用异常(anomalies)
  \end{itemize}
\end{frame}
%%%%%%%%%%%%%%%
\begin{frame}{数据一致性问题举例 (I)}
  \fig{width = 0.60\textwidth}{figures/data-inconsistency-comment-reordering.pdf}
  {社交网络中, 消息-评论乱序 \citeinbeamer{Lloyd}{CACM}{14}.}
\end{frame}
%%%%%%%%%%%%%%%
\begin{frame}{数据一致性问题举例 (II)}
  \begin{figure}[h!]
    \centering
    \begin{adjustbox}{max totalsize = {0.65\textwidth}{0.65\textheight}, center}
      %        File: data-inconsistency-rmw.tex
%     Created: Thu Oct 08 10:00 AM 2015 C
% Last Change: Thu Oct 08 10:00 AM 2015 C
%
\begin{tikzpicture}
  \uncover<1->{
  \begin{scope}
    % phone
    \node (phone) [] at (0,0) {\includegraphics[scale = 0.4]{figures/phone-icon.png}};
    % tablet
    \node (tablet) [below left = 4.0cm and 3.0cm of phone] {\includegraphics[scale = 
    0.55]{figures/tablet-icon.jpg}};
    % pc
    \node (pc) [below right = 4.0cm and 3.0cm of phone] {\includegraphics[scale = 
    0.60]{figures/pc-icon}};
  \end{scope}
  }

    % write from pc 
  \uncover<2->{
    \begin{scope}[<->, blue, line width = 6pt, font = \huge]
      % person at pc
      \only<2>{
      \node (person-pc) [above left = -1.0cm and -2.0cm of pc] {\includegraphics[scale = 
      0.30]{figures/person-icon}};
      }
      \draw [] (pc.west) to node [midway, sloped, above] {\textbf{1.} update file $f$} 
      (tablet.east);
      \draw [red, loosely dashed] (pc.north) to [out = 90, in = -20] node [midway, sloped, font = 
      \Huge, scale = 2]{$\times$} (phone.east);
      \draw (pc) to [loop, out = 60, in = -10, looseness = 3] node [midway, above, sloped] 
      {\textbf{1.} update file $f$} (pc);
    \end{scope}
   }


    % read from phone
   \uncover<3->{
    \begin{scope}[<->, brown, line width = 6pt, font = \huge]

      % person at phone
      \node (person-phone) [above left = -1.0cm and -2.0cm of phone] {\includegraphics[scale = 
      0.30]{figures/person-icon}};

      \draw (phone) to [loop, out = 180, in = -100, looseness = 3] node [midway, below, sloped] 
      {\textbf{2.} read file $f$} (phone);   
    \end{scope}

    % update lost
    \node (inconsistency) [font = \huge, right = of person-phone, red] {\bf Update Lost!};
  }
\end{tikzpicture}

    \end{adjustbox}
    \caption{多设备文件共享时, 更新丢失 ($\#N = 3, \#W = 2, \#R = 1$).}
  \end{figure}
\end{frame}
%%%%%%%%%%%%%%%
\begin{frame}{数据一致性问题研究的历史阶段}
  \fignocaption{width = 0.30\textwidth}{figures/taste-of-past.jpg}

  \todo{图: ps}
\end{frame}
%%%%%%%%%%%%%%%
\begin{frame}{数据一致性问题研究的历史阶段 (I; 1970s)}
  \fignocaption{width = 0.40\textwidth}{figures/notes-on-distributed-databases.png}

  \begin{center}
	Tradeoff availability for consistency {\scriptsize (and simplicity)}.
  \end{center}
\end{frame}
%%%%%%%%%%%%%%%
\begin{frame}{数据一致性问题研究的历史阶段 (II; Middle 1990s)}
  Eventual {\scriptsize (weak)} consistency \citeinbeamer{Terry}{PDIS}{94}, \citeinbeamer{Terry}{SOSP}{95} driven by mobile computing.
  
  \fignocaption{width = 0.40\textwidth}{figures/bayou-paper.png}

  \begin{center}
	Tradeoff consistency for availability.
  \end{center}
\end{frame}
%%%%%%%%%%%%%%%
\begin{frame}{数据一致性问题研究的历史阶段 (III; Early 2000s)}
  \fignocaption{width = 0.50\textwidth}{figures/cap-theorem.pdf}
  \todo{图: 重绘}

  CAP定理: 分布式系统\question{无法}同时满足强一致性, 可用性和分区容错性.
  \citeinbeamer{Brewer}{PODC}{00} \citeinbeamer{Gilbert}{SIGACT}{02}
\end{frame}
%%%%%%%%%%%%%%%
\begin{frame}{数据一致性问题研究的历史阶段 (IV; Late 2000s)}
  \fignocaption{width = 0.30\textwidth}{figures/amazon-dynamo-logo.jpg}

  \begin{columns}
	\column{0.50\textwidth}
	  %\fignocaption{width = 0.50\textwidth}{figures/}
	\column{0.50\textwidth}
	  %\fignocaption{width = 0.50\textwidth}{figures/}
  \end{columns}
\end{frame}
%%%%%%%%%%%%%%%
\begin{frame}{数据一致性问题研究的历史阶段 (V; 2010s)}
  \fignocaption{width = 0.65\textwidth}{figures/pacelc-tradeoff.pdf}
\end{frame}
%%%%%%%%%%%%%%%
\begin{frame}{数据一致性问题研究的发展趋势}
  云计算凸显应用价值观

  binary $\Rightarrow$ spectrum

  tunable consistency

  \fignocaption{width = 0.30\textwidth}{figures/sla-dashboard.jpg}
\end{frame}
%%%%%%%%%%%%%%%
\begin{frame}{数据一致性问题研究的发展趋势}
  \fignocaption{width = 0.40\textwidth}{figures/sla.jpg}

  更灵活的数据一致性模型:
  \begin{enumerate}
    \setlength{\itemsep}{5pt}
    \item 多样化, 可定制 \citeinbeamer{Terry}{CACM}{13}
    \uncover<2->{
      \begin{description}
        \item[多样化:] 
          \begin{itemize}
            \item 一致性族: causality; read-your-writes {\scriptsize (RYW)}
            \item 参数调节: 提供``有限度''的不一致 \citeinbeamer{Yu}{TOCS}{02}
          \end{itemize}
        \item[可定制:] 混合使用, 运行时可变
      \end{description}
    }
  \item 精细化, 可度量 \citeinbeamer{Bailis}{VLDB}{12}
    \uncover<3->{
      \begin{description}
        \item[精细化:] ``在大多数情况下, 访问到一致数据''
        \item[可度量:] 量化系统执行, 后验系统对一致性的满足程度 
      \end{description}
    }
  \end{enumerate}
\end{frame}
%%%%%%%%%%%%%%%
%%%%%%%%%%%%%%%
%%%%%%%%%%%%%%%
%%%%%%%%%%%%%%%
