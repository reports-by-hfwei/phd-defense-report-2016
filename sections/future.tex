%%%%%%%%%%%%%%%%%%%%%%%%%%%%%%%%%%%%%%%%%%%%%%%%%%%%%%%%%%%%%%%%%%%%%%%%%%%%%%%	
\section{总结展望}

%%%%%%%%%%%%%%%
\begin{frame}{本文贡献}
  \begin{description}
	\item[理念:] 提出\idea{}的\\分布数据一致性问题研究理念
	\item[框架:] 提出包含``一个基础、三个维度''的技术框架 
	  \pause
	  \vspace{0.20cm}
	\item[VPC:] 验证 PRAM 一致性 \hfill \textcolor{brown}{\scriptsize [``精细化, 可度量'']}
	  \begin{itemize}
		\item 问题复杂度分析 + 算法设计
		\item 较早系统研究弱一致性模型的验证问题
	  \end{itemize}
	  \pause
	\item[PA2AM:] 量化 2-Atomicity 协议 \hfill \textcolor{brown}{\scriptsize [``精细化, 可度量'']} 
	  \begin{itemize}
		\item PA2AM 一致性: ``确定性有界'' + 概率
		\item PA2AM 算法量化分析: 针对 atomicity
	  \end{itemize}
	  \pause
	\item[RVSI:] 多样化 Snapshot Isolation \hfill \textcolor{brown}{\scriptsize [``多样化, 可调节'']}
	  \begin{itemize}
		\item 可控制``异常''程度
		\item 可调节一致性强度
	  \end{itemize}
  \end{description}
\end{frame}
%%%%%%%%%%%%%%%
\begin{frame}{未来工作}
  具体问题:
  \begin{enumerate}
	\setlength{\itemsep}{5pt}
	\item 验证一致性模型 \textcolor{brown}{\small (遵循 VPC 工作思路)}
	  \begin{itemize}
		\item VHC {\small (HC: Hybrid Consistency)} 问题 \citeinbeamer{Attiya}{SICOMP}{98}
		\item $k$-AV {\small ($k$-Atomicity Verification)} 问题 \citeinbeamer{PODC}{Golab}{15}
	  \end{itemize}
	  \pause
	\item ``多写模型''下的 atomic 寄存器 \textcolor{brown}{\small (扩展 PA2AM 工作)}
      \begin{description}
		\setlength{\itemsep}{3pt}
        \item[低延迟:] 读操作只需一轮网络通信
        \item[可能性问题:] 是否存在低延迟 ($k$-)atomicity 算法?
		\item[尽可能强:] 如何定义并量化 $p$AM ($p$: probabilistic)?
      \end{description}
  \end{enumerate}
\end{frame}
%%%%%%%%%%%%%%%
\begin{frame}{未来工作}
  宏观方向:
  \begin{enumerate}
	\setlength{\itemsep}{8pt}
	\item 考虑更丰富的数据类型 {\small (set, queue 等)} \citeinbeamer{Burckhardt}{POPL}{14}
	\item ``多样化, 可调节'' 的理论基础
	  \begin{itemize}
		\item 多样化一致性模型的复杂度
		\item 编程方法
	  \end{itemize}
  \end{enumerate}
\end{frame}
%%%%%%%%%%%%%%%
\begin{frame}{发表论文}

  {\small
  \begin{itemize}
	\item \textcolor{blue}{[TC'16]} {\bf Hengfeng Wei}, Yu Huang, Jian Lu. 
	  Probabilistically-Atomic 2-Atomicity: Enabling Almost Strong Consistency in Distributed Storage Systems. 
	  In {\it IEEE Trans. Comput.}, xx(x):x--x, PrePrints, 2016.
	\item \textcolor{blue}{[TPDS'16]} {\bf Hengfeng Wei}, Marzio De Biasi, Yu Huang, Jiannong Cao, Jian Lu. 
	  Verifying Pipelined-RAM Consistency over Read/Write Traces of Data Replicas.
	  In {\it IEEE Trans. Parallel Distrib. Syst.}, 27(5):1511--1523, 2016.
	\item \textcolor{blue}{[PerCom'12]} {\bf Hengfeng Wei}, Yu Huang, Jiannong Cao, Xiaoxing Ma, Jian Lu. 
	  Formal Specification and Runtime Detection of Temporal Properties for Asynchronous Context. 
	  In {\it Proceedings of the 10th IEEE International Conference on Pervasive Computing and Communications},
	  pages 30--38, 2012.
	\item \textcolor{blue!50!gray}{[WiP for VLDB'17]} {\bf Hengfeng Wei}, Yu Huang, Jian Lu.
	  Relaxed Version Snapshot Isolation in Distributed Transactional Key-Value Stores. 
  \end{itemize}
  }
\end{frame}
%%%%%%%%%%%%%%%
\begin{frame}{致谢}
  \begin{itemize}
	\setlength{\itemsep}{15pt}
	\item 导师: 吕建教授、黄宇副教授
	\item 论文评审与答辩老师
	\item 答辩秘书: 余萍副教授
	\item 软件所全体师生
  \end{itemize}
\end{frame}
%%%%%%%%%%%%%%%
\begin{frame}[noframenumbering]
  \fignocaption{width = 0.20\textwidth}{figures/qa.png}
  \vspace{-0.8cm}
  \begin{center}
    \textcolor{blue}{\bf \large hengxin0912@gmail.com}
  \end{center}
  \vspace{-0.5cm}
  \fignocaption{width = 0.50\textwidth}{figures/thankyou.jpg}
\end{frame}
%%%%%%%%%%%%%%%
