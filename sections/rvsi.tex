%%%%%%%%%%%%%%%%%%%%%%%%%%%%%%%%%%%%%%%%	
\subsection{RVSI: Snapshot Isolation 一致性弱化与维护}

\newcommand{\chameleon}{\textsc{Chameleon}}
\newcommand{\konebv}{$k_1$-BV}
\newcommand{\ktwofv}{$k_2$-FV}
\newcommand{\kthreesv}{$k_3$-SV}
%%%%%%%%%%%%%%%
\begin{frame}{RVSI 工作在技术框架中的位置}
  \fig{width = 0.50\textwidth}{figures/3d-framework-rvsi.pdf}
	{RVSI --- Snapshot Isolation 一致性弱化与维护.}
\end{frame}
%%%%%%%%%%%%%%%
\begin{frame}{研究动机}
  \question{问题: 为什么要提出 Relaxed Version Snapshot Isolation {\small (RVSI)} 一致性?}
  \vspace{0.20cm}

  \begin{description}
    \setlength{\itemsep}{5pt}
    \item[分布式事务:]
      \begin{itemize}
        \item ``all-or-none'' 语义
        \item 受到分布式存储系统的关注 \citeinbeamer{Cassandra}{CASSANDRA-ISSUE-7056}{14}
      \end{itemize}
    \item[弱一致性:] 
        PCSI \citeinbeamer{Elnikety}{SRDS}{05} 
		  \textcolor{red}{SI} \citeinbeamer{Lin}{TODS}{09} \\
          PSI \citeinbeamer{Sovran}{SOSP}{11} NMSI \citeinbeamer{Ardekani}{SRDS}{13} 
    \pause
	\item[\textcolor{red}{异常控制:}] 容忍``有限度的''异常 \citeinbeamer{Yu}{TOCS}{02}
	\item[\textcolor{red}{可调节:}] 
      \begin{itemize}
        \item 不同应用对一致性需求不同 \citeinbeamer{Terry}{CACM}{13}
        \item 运行时决定 \citeinbeamer{Terry}{SOSP\&TR}{13}
      \end{itemize}
  \end{description}
\end{frame}
%%%%%%%%%%%%%%%
\begin{frame}{RVSI 定义}
  RVSI 定义原则:
  \begin{itemize}
    \item 参数 $k_1, k_2, k_3$ 控制``异常''程度
    \item $\text{RC} \supset \text{RVSI}(k_1, k_2, k_3) \supset \text{SI}$
    \item $\text{RVSI}(\infty,\infty,\infty) = \text{RC}; \qquad \text{RVSI}(1,0,\ast) = \text{SI}$
  \end{itemize}

  \vspace{0.20cm}

  \begin{cdef}[RVSI: Relaxed Version Snapshot Isolation]
    \begin{description}
      \item[单变量读 $\texttt{read}(x)$:] \hfill 
        \begin{enumerate}
		  \item 允许读 $\le k_1$ 陈旧值 (\konebv{})
		  \item 允许读 $\le k_2$ 并发更新 (\ktwofv{})
        \end{enumerate}
      \item[多变量读 $\texttt{read}(x), \texttt{read}{(y)}$:] \hfill
        \begin{enumerate}
          \setcounter{enumi}{2}
		\item $\textrm{dist}(\textsl{snap}{(x)},\textsl{snap}{(y)}) \le k_3$
        \end{enumerate}
    \end{description}
  \end{cdef}
\end{frame}
%%%%%%%%%%%%%%%
\begin{frame}{\chameleon{} 分布式事务键值存储原型系统设计}
  \begin{description}
	\item[系统架构:] 阿里云\footnote{阿里云: \url{https://www.aliyun.com/}.} 
	  多数据中心 {\small ($9 = 3 \times 3$)}
	\item<2->[数据分区:] 同一数据中心
	\item<2->[数据副本:] 跨数据中心; 主从结构
  \end{description}

  \fignocaption{width = 0.55\textwidth}{figures/chameleon-arch.pdf}
\end{frame}
%%%%%%%%%%%%%%%
\begin{frame}{\chameleon{} 分布式事务键值存储原型系统设计}
  \begin{description}
	\item[系统组件:] 客户端库 + 数据中心
	\item<2->[数据分区:] 分布式事务原子提交协议 {\small (2PC)}
	\item<2->[数据副本:] 懒惰复制 {\small (lazy replication)} 协议
  \end{description}

  \fignocaption{width = 0.42\textwidth}{figures/chameleon-framework.pdf}
\end{frame}
%%%%%%%%%%%%%%%
\begin{frame}{RVSI 维护算法}
  \[
    \textcolor{blue}{\text{RC} \supset \text{RVSI}(k_1, k_2, k_3) \supset \text{SI}}
  \]

  \vspace{0.10cm}

  RVSI 维护算法:
  \begin{itemize}
    \item 以分布式 RC 和 SI 协议 为基础
    \item 事务执行时, 添加 RVSI ``版本约束'' ($k_1, k_2, k_3$ 相关不等式)
    \item 事务提交时, 检查 RVSI ``版本约束''
  \end{itemize}

  % \fignocaption{width = 0.50\textwidth}{figures/chameleon-build-passing.png}
  \textcolor{red}{\small \url{https://github.com/hengxin/chameleon-transactional-kvstore}}
\end{frame}
%%%%%%%%%%%%%%%
\begin{frame}{RVSI 实验评估}
  \begin{table}[]
  \renewcommand{\arraystretch}{1.1}
  \centering
  \caption{事务负载参数表.\protected\\(\textcolor{blue}
	{评估目标: RVSI 对事务中止率的影响})}
  \resizebox{\textwidth}{!}{%
  \begin{tabular}{|c||c|c|c|}
	\hline
	{\bfseries Parameter}   & {\bfseries F(ixed)/V(ariable)/R(andom)}	
	& {\bfseries Value}		& {\bfseries Explanation}
	\\ \hline  \hline
	\#keys  				& F		& 5  				&  	size of keyspace
	\\ \hline
	\cellcolor{brown}mpl	& \textcolor{red}{V}		& 5, 10, 15, 20, 25, 30
	& \innercell{c}{multiprogramming level: \\ number of concurrent clients} \\ \hline
	\#txs/client					& F		& 1000 						
	& \innercell{c}{number of txs per client}
	\\ \hline
	\#ops/tx					& R		& $\sim$ Binomial(20, 0.5)	
	&  \innercell{c}{number of operations per tx}
	\\ \hline
	\cellcolor{brown}rwRatio & \textcolor{red}{V} 
	  & 1:2, 1:1, 4:1 & {\#reads}/{\#writes}
	\\ \hline
	zipfExponent			& F		& 1		& parameter for Zipfian distribution
	\\ \hline  \hline
	\cellcolor{brown}$(k_1, k_2, k_3)$		& \textcolor{red}{V}
		&  \innercell{c}{(1,0,0) \\ (1,0,2) (1,1,0) \\ (2,0,0) (2,1,2) (2,2,1)}	
		&  for \konebv{}, \ktwofv{}, and \kthreesv{}
	\\ \hline  \hline
	minInterval				& F		& 0ms		& minimum inter-transactions time
	\\ \hline
	maxInterval				& F		& 10ms		& maximum inter-transactions time
	\\ \hline
	meanInterval			& R		& 5ms		
	& \innercell{c}{mean inter-transactions time \\ for exponential distribution}
	\\ \hline
  \end{tabular}
  }
\end{table}

\end{frame}
%%%%%%%%%%%%%%%
\begin{frame}{RVSI 实验评估}
  \fig{width = 0.70\textwidth}{figures/rvsi-rw4-abort-rates.pdf}
	{读频繁 (rwRatio = 4:1) 负载下 RVSI 对事务中止率的影响.}

  \begin{description}
	\item[\textcolor{blue}{wcf-aborted:}] 
	  无显著变化 {\small ($\text{wcf}(1,0,0) = 0.184733$)}
	\item[\textcolor{red}{vc-aborted:}] 显著减少 
	  {\small ($\text{vc}(1,0,0) = 0.204733;
	  \text{vc}(1,1,0) = 0.066433;
	  \text{vc}(2,2,1) = 0.002033$)}
  \end{description}
\end{frame}
%%%%%%%%%%%%%%%
\begin{frame}{RVSI 实验评估}
  \begin{columns}
	\column{0.48\textwidth}
	  \fig{width = 1.00\textwidth}{figures/rvsi-rw05-abort-rates.pdf}
		{写频繁 (rwRatio = 1:2) 负载下 RVSI 对事务中止率的影响.}
	\column{0.48\textwidth}
	  \fig{width = 1.00\textwidth}{figures/rvsi-rw1-abort-rates.pdf}
		{读写相当 (rwRatio = 1:1) 负载下 RVSI 对事务中止率的影响.}
  \end{columns}
\end{frame}
%%%%%%%%%%%%%%%
\begin{frame}{RVSI 的意义}
  \mdf{red}{blue}{RVSI 对事务中止率的影响}{teal}{
	\begin{enumerate}
	  \item 适当放松事务对 RVSI 版本规约的要求可降低事务中止率
	  \item RVSI 能否``显著''降低事务中止率与负载类型相关
    \end{enumerate}
  }
\end{frame}
%%%%%%%%%%%%%%%
