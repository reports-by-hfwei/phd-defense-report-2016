%%%%%%%%%%%%%%%%%%%%%%%%%%%%%%%%%%%%%%%%
\subsection{VPC: Pipelined-RAM 一致性验证}

\newcommand{\pram}{Pipelined-RAM}
\newcommand{\PRAM}{PRAM}
\newcommand{\vpc}[1]{\ifthenelse{\isempty{#1}{}}{\textsf{VPC}}{\textsf{VPC-\MakeUppercase{#1}}}} 
\newcommand{\npc}{$\sf{NP}$-complete}
\newcommand{\npcn}{$\sf{NP}$-completeness}
\newcommand{\rwclosure}{\textsc{RW-Closure}}
\newcommand{\readcentric}{\textsc{Read-Centric}}
%%%%%%%%%%%%%%%
\begin{frame}{VPC 工作在技术框架中的位置}
  \fig{width = 0.50\textwidth}{figures/3d-framework-vpc.pdf}{VPC --- \pram{} {\small (\PRAM{})} 一致性验证.}
\end{frame}
%%%%%%%%%%%%%%%
\begin{frame}{VPC 问题定义}
  \begin{cdef}[VPC: Verifying PRAM Consistency]
    VPC 判定问题:
	\vspace{8pt}
    \begin{description}
	  \setlength{\itemsep}{8pt}
      \item[实例:] 系统执行 {\small (execution $e$)}
      \item[问题:] 该执行 $e$ 是否满足 PRAM 一致性模型 {\small ($\mathcal{C}$)}? 
		\[
		  e \in \mathcal{C} \Rightarrow \set{0,1}?
		\]
    \end{description}    
  \end{cdef}
\end{frame}
%%%%%%%%%%%%%%%
\begin{frame}{VPC 问题定义}
  \begin{cdef}[系统执行]
	系统执行 ($e$) $\triangleq$ $\{h_p \mid h_p: \text{进程 } \;p \text{ 上的读写操作序列}\}$

	\vspace{0.30cm}
	\textcolor{blue}{规模 $n$:} 系统执行中操作的总数
  \end{cdef}

  \fignocaption{width = 0.50\textwidth}{figures/execution-example.pdf}
\end{frame}
%%%%%%%%%%%%%%%
\begin{frame}{VPC 问题定义}
  \begin{cdef}[\PRAM{} 一致性模型]
	\begin{center}
	  系统执行 $e$ 满足 \PRAM{} 一致性 \\
	  $\iff$\\
	  $\forall p:$ $p$ 上\textcolor{blue}{所有操作}与其它进程上所有\textcolor{blue}{写操作}存在\textcolor{red}{合法}调度.
	\end{center}
  \end{cdef}

  \fignocaption{width = 0.50\textwidth}{figures/execution-example.pdf}

  \pause
  \vspace{-0.80cm}

  \begin{gather*}
	p_{0}: Wf2\; Wf1\; Wz2\; Wz1\; Wy2\; Wy1\; \textcolor{brown}{Rf1} 
	Wx5\; Wx3\; Wx2\; Wc1\; \textcolor{brown}{Rc1} \\
	\textcolor{brown}{Rz1} \textcolor{brown}{Ry1}
	Wa1\; \textcolor{brown}{Ra1} Wb1\; \textcolor{brown}{Rb1} \textcolor{brown}{Rx2}
  \end{gather*}
\end{frame}
%%%%%%%%%%%%%%%
\begin{frame}{研究动机}
  \question{问题: 为什么要验证 \PRAM{} 一致性?}
  \vspace{0.10cm}

  \begin{description}
    \setlength{\itemsep}{10pt}
	\item[验证:] 用户与商家就数据一致性签订 SLA {\small (Service Level Agreement)} 协议
	  \citeinbeamer{Amazon}{SOSP}{07} \citeinbeamer{Golab}{PODC}{11}
	  \vspace{2pt}
	  \begin{itemize}
		\item \textcolor{teal}{[商家]} 系统测试手段之一
		\item \textcolor{teal}{[用户]} 确认系统是否提供了其所声称的数据一致性 
	  \end{itemize}
	\pause
	\item[PRAM:] 存储系统常提供``会话'' {\small (session)} 一致性\\
      \citeinbeamer{Saito}{CSUR}{05} \citeinbeamer{Terry}{CACM}{13}
	  \vspace{2pt}
      \begin{itemize}
		\item 包含了弱一致性的诸多变体 
		\item 近似于 PRAM 一致性 \citeinbeamer{Brzezi$\acute{\text{n}}$ski}{PDP}{04} \citeinbeamer{Bailis}{VLDB}{13}
      \end{itemize}
  \end{description}
\end{frame}
%%%%%%%%%%%%%%%
\begin{frame}{VPC 问题分类}
  \begin{table}[!t]
    \centering
	\caption{VPC 问题的四种变体 (按``执行''的类型) 及复杂度分析 ($\textcolor{red}{[\ast]}$\textcolor{red}{: 本文工作}).}
	\renewcommand\arraystretch{1.2}
    \begin{tabular}{|c|c|c|}
      \hline
      & \it (S)ingle variable  & \it (M)ultiple variables
      \\ \hline
	  {\it write (D)uplicate values} &
	  \innercell{c}{VPC-SD \\ \uncover<2->{\textcolor{red}{\small (\npc{}) $[\ast]$}}} &
	  \innercell{c}{VPC-MD \\ \uncover<2->{\textcolor{red}{\small (\npc{}) $[\ast]$}}}
      \\ \hline
	  \only<1-2>{\it write (U)nique value}\only<3>{\cellcolor{brown!80}{\it write (U)nique value}} &
	  \innercell{c}{VPC-SU \\ \uncover<2->{\textcolor{blue}{\small (P) \citeinbeamer{Golab}{PODC}{11}}}} &
	  \innercell{c}{VPC-MU \\ \uncover<2->{\textcolor{red}{\small (P) $[\ast]$}}}
      \\ \hline
    \end{tabular}
  \end{table}

  \vspace{10pt}
  
  \uncover<3->{\textcolor{brown}{\centerline{Read-mapping \citeinbeamer{Gibbons}{SICOMP}{97}: $\forall r, f(r) = w$.}}}
\end{frame}
%%%%%%%%%%%%%%%
\begin{frame}{VPC-SD (VPC-MD) 是 \npc{} 问题}
  多项式规约: 从 \textsc{Unary 3-Partition} 到 \vpc{sd}.

  \fig{width = 0.40\textwidth}{figures/vpcsd-npc.pdf}{对应于 \textsc{Unary 3-Partition} 实例 $A = \{2,2,1,1,1,1\}, m = 2, B = 4$ 的 VPC 执行.} 
\end{frame}
%%%%%%%%%%%%%%%
\begin{frame}{VPC-MU 的多项式算法 \rwclosure{}}
  \begin{figure}[h!]
    \centering
    \begin{adjustbox}{max totalsize = {0.60\textwidth}{1.00\textheight}, center}
	  \input{tikz-in-beamer/rw-closure-example}
    \end{adjustbox}
	\caption{\rwclosure{} 算法示例: 在传递闭包之上迭代应用 $w'wr$ 规则.}
  \end{figure}

  \uncover<4->{\fig{width = 0.25\textwidth}{figures/wprimewr-order.pdf}{$w'wr$ 规则.}}
\end{frame}
%%%%%%%%%%%%%%%
\begin{frame}{VPC-MU 的多项式算法 \rwclosure{}}
  \begin{ctheorem}[\rwclosure{} 算法正确性]
	\begin{center}
	  \vpc{mu} 实例满足 \PRAM{} 一致性\\
		$\iff$\\
	  \rwclosure{} 算法所得图是 DAG 图.
	\end{center}
  \end{ctheorem}

  \pause
  \vspace{0.80cm}

  \rwclosure{} 算法复杂度: 
  \[
    \underbrace{O(n^2)}_{\textrm{\#loops}} \quad\cdot
	\underbrace{O(n^3)}_{\textrm{transitive closure}}  = O(n^5)
  \]
\end{frame}
%%%%%%%%%%%%%%%
\begin{frame}{VPC-MU 的多项式算法 \readcentric{}}
  \rwclosure{} 算法的缺点:
  \begin{itemize}
	\item 在全图上应用 $w'wr$ 规则
	\item 应用 $w'wr$ 规则无特定顺序
  \end{itemize}

  \pause
  \vspace{0.50cm}

  \readcentric{} 算法要点:
  \begin{itemize}
	\item \textcolor{blue}{增量式}调度每个读操作
	\item 在读操作诱导的\textcolor{blue}{局部子图}上按\textcolor{blue}{逆拓扑序}应用 $w'wr$ 规则
  \end{itemize}

  \pause
  \vspace{0.60cm}
  \rwclosure{} 算法复杂度: 
  \[
    \underbrace{O(n)}_{\textrm{iterations}} \cdot
	\underbrace{O(n^3)}_{\textsc{Topo-Schedule}} = O(n^4)
  \]
\end{frame}
%%%%%%%%%%%%%%%
\begin{frame}{实验评估}
  实验目的:
  \begin{enumerate}
	\item 考察 \readcentric{} 算法的实际效率 
	  \textcolor{blue}{\small ({\it vs.} 渐近时间复杂度)}
	\item 对比 \readcentric{} 算法与 \rwclosure{} 算法的效率
  \end{enumerate}

  \pause
  \vspace{0.50cm}

  两类负载:
  \begin{enumerate}
	\item 随机生成的系统执行
	\item 满足 \PRAM{} 一致性的系统执行 \textcolor{red}{\small ($\approx$ 最坏情况输入)}
  \end{enumerate}
\end{frame}
%%%%%%%%%%%%%%%
\begin{frame}{实验评估}
  \begin{figure}[t]
	\centering
	\begin{subfigure}[t]{0.50\textwidth}
	  \includegraphics[width = 0.80\textwidth]{figures/vpc-random-cmp.pdf}
	\end{subfigure}%
	~
	\begin{subfigure}[t]{0.50\textwidth}
	  \includegraphics[width = 0.80\textwidth]{figures/vpc-valid-cmp.pdf}
	\end{subfigure}
	\caption{\rwclosure{} 算法与 \readcentric{} 算法在
	\textcolor{blue!80}{ (左) 随机生成}的执行及
	\textcolor{red!80}{ (右) 满足 \PRAM{} 一致性}的执行上的运行时间。}
  \end{figure}

  \pause
  \begin{center}
	\textcolor{red}{(右)} 20个进程、8,000 个操作: 

	\readcentric{} 可获得 694 倍加速.
  \end{center}
\end{frame}
%%%%%%%%%%%%%%%
\begin{frame}{实验评估}
  \fig{width = 0.50\textwidth}{figures/vpc-scalability-more.pdf}
  {\readcentric{} 算法在满足 \PRAM{} 一致性的执行上的运行时间.}

  \begin{description}
	\centering
	\item[\readcentric{}:] 20个进程、60,000个操作 < 600s
	\item[\rwclosure{}:] 20个进程、8,000个操作 > 3,000s
  \end{description}
\end{frame}
%%%%%%%%%%%%%%%
\begin{frame}{VPC 的意义}
  \mdf{red}{blue}{对 VPC 问题的系统研究}{teal}{
	\begin{enumerate}
	  \item \readcentric{} 算法可用于测试系统是否正确实现了 \PRAM{} 一致性模型
	  \item \npcn{} 结果有助于理解弱一致性模型的复杂度
	\end{enumerate}
  }
\end{frame}
%%%%%%%%%%%%%%%
