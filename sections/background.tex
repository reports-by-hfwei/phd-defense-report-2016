\section{研究背景}

%%%%%%%%%%%%%%%
\begin{frame}{分布式应用}
  \fignocaption{width = 0.45\textwidth}{figures/sina-weibo-world-map.pdf}
  \vspace{0.50cm}

  \begin{columns}
	\column{0.40\textwidth}
	新浪微博社交应用~\footnotemark:
	\begin{itemize}
	  \item 日均用户近一亿名
	  \item 日均消息近一亿条
	\end{itemize}

	\column{0.50\textwidth}
	底层数据服务系统特性需求~\term{$H^{3}L$}: 
	\begin{itemize}
	  \item 低延迟, 高可用性 (4个9~\footnotemark)
	  \item 高容错性, 高可扩展性
	\end{itemize}
  \end{columns}
  
  \footnotetext[1]{2015第三季度; 数据来自 \href{https://www.chinainternetwatch.com/15740/weibo-q3-2015/}{China Internet Watch}.}
  \footnotetext[2]{数据来自 \href{http://www.infoq.com/cn/articles/weibo-platform-availability-9999}{InfoQ}.}
\end{frame}
%%%%%%%%%%%%%%%
\begin{frame}{分布数据}
  \graphicspath{{tikz-in-beamer/}}
  \begin{figure}[h!]
    \centering
    \begin{adjustbox}{max totalsize = {0.50\textwidth}{1.00\textheight}, center}
	  \input{tikz-in-beamer/distributed-data-overlay-for-background}
    \end{adjustbox}
  \end{figure}

  \textcolor{red}{分布数据 \term{distributed data}:}
  \begin{enumerate}
	\item<2-> 分区 \term{partition}: 水平扩展
	\item<3-> 副本 \term{replication}: 就近访问, 容灾备份
  \end{enumerate}
\end{frame}
%%%%%%%%%%%%%%%
% \begin{frame}{分布共享数据服务}
%   \fignocaption{width = 0.55\textwidth}{figures/sina-weibo-redis.pdf}
% 
%   \begin{center}
% 	分布共享数据服务 \term{中间件}
% 
% 	\term{Distributed Shared Data Service}
% 	\vspace{0.30cm}
% 
% 	屏蔽底层数据分布性\; 提供共享数据抽象\; 简化上层应用开发
%   \end{center}
% \end{frame}
% %%%%%%%%%%%%%%%
% \begin{frame}{分布共享数据服务典型应用 (I)}
%   \fig{width = 0.55\textwidth}{figures/dsss.pdf}
%   {分布式存储系统 (\textcolor{blue}{\scriptsize 开源 [左]} \& \textcolor{red}{\scriptsize 商用 [右]}).}
% 
%   % \begin{description}
%   %   \item[低延迟:] 就近访问副本数据
%   %   \item[高可用性, 高容错性:] 备份容灾 
%   % \end{description}
% \end{frame}
% %%%%%%%%%%%%%%%
% \begin{frame}{分布共享数据服务典型应用 (II)}
%   \fig{width = 0.75\textwidth}{figures/file-share.pdf}
%   {个人多设备文件共享 {(\textcolor{blue}{\scriptsize [基于云] C/S 结构 [左]} \& 
%   \textcolor{red}{\scriptsize P2P 结构 [右]}).}}
% 
%   \begin{description}
%     \item[功能需求:] 文件副本 \citeinbeamer{Strauss}{MIT Thesis}{10}
%     \item[网络断连:] 备份容灾; 离线可用
%   \end{description}
% \end{frame}
%%%%%%%%%%%%%%%
% \begin{frame}{分布副本数据的典型应用 (三)}
%   \fig{width = 0.80\textwidth}{figures/coordination.pdf}
%   {分布式协同应用(上)及服务(下).}
% 
%   分布式协同服务需求 \citeinbeamer{Burrows}{OSDI}{06} vs. ``副本技术'':
%   \begin{description}
%     \item[高性能:] 低延迟就近``读''副本 \citeinbeamer{Yahoo!}{USENIXATC}{10}
%     \item[高可靠性:] 避免单点协同故障
%   \end{description}
% \end{frame}
%%%%%%%%%%%%%%%
