\documentclass[shownotes, xcolor = table]{beamer}

\usepackage{xeCJK}
\usepackage{zhnumber}	% counters in Chinese
\usepackage{fontspec}
\usepackage{comment}
\usepackage{xifthen}

\usetheme{CambridgeUS} % try Madrid
\usecolortheme{beaver} % try beaver, dolphin, seahorse
\usefonttheme[onlymath]{serif} % try "professionalfonts"
\setCJKmainfont{Microsoft YaHei} % try SimSun

\usepackage{amsmath, amsfonts, amssymb, mathtools, pifont}
\newcommand{\cmark}{\ding{51}}%
\newcommand{\xmark}{\ding{55}}%
\def\checkmark{\tikz\fill[scale=0.5](0,.35) -- (.25,0) -- (1,.7) -- (.25,.15) -- cycle;} 

\usepackage{graphicx, subcaption}
\usepackage{adjustbox}

\usepackage[framemethod=TikZ]{mdframed}

\usepackage[normalem]{ulem} % strike through text
\newcommand{\soutthick}[1]{%
    \renewcommand{\ULthickness}{2.0pt}%
       \sout{#1}%
    \renewcommand{\ULthickness}{.4pt}% Resetting to ulem default
}

% \usepackage{biblatex}
% \addbibresource{phd-defense-report.bib}

\setbeamersize{text margin left = 2em, text margin right = 1em}
\setbeamercolor{footnote mark}{fg = teal}
\setbeamertemplate{itemize items}[default]
\setbeamertemplate{enumerate items}[default]

% \setlength{\leftmarginii}{5pt}

\usepackage{tikz}
\usepackage{pgfplots}
\usetikzlibrary{arrows.meta, shapes, positioning, calc, backgrounds, fit}

\theoremstyle{plain}
\newtheorem{cdef}{定义}[section]
\newtheorem{ctheorem}{定理}[section]
\newtheorem{cquestion}{问题:}[section]
\newtheorem{cobservation}{观察:}[section]
\theoremstyle{proof}
\newtheorem{cproof}{证明}[section]

% for tables
\usepackage{multirow}
\newcommand{\innercell}[2]{\begin{tabular}{@{}#1@{}}#2\end{tabular}}
% \usepackage{colortbl}
\usepackage{hhline}
%%%%%%%%%%%%%% for appendix %%%%%%%%%%%%%%%%
% http://www-ljk.imag.fr/membres/Jerome.Lelong/latex/appendixnumberbeamer.sty
% Reference: http://tex.stackexchange.com/questions/2541/beamer-frame-numbering-in-appendix
\usepackage{appendixnumberbeamer}
% Add total frame count to slides, optional. From Stefan,
% http://www.latex-community.org/forum/viewtopic.php?f=4&t=2173
\expandafter\def\expandafter\insertshorttitle\expandafter{%
  \insertshorttitle\hfill\insertframenumber\,/\,\inserttotalframenumber}
%%%%%%%%%%%%%% for appendix %%%%%%%%%%%%%%%%

% for fig without caption: #1: width/size; #2: fig file
\newcommand{\fignocaption}[2]{
  \begin{figure}[htp]
    \centering
    \includegraphics[#1]{#2}
  \end{figure}
}

% for fig without caption: #1: width/size; #2: fig file; #3: fig caption
\newcommand{\fig}[3]{
  \begin{figure}[htp]
    \centering
      \includegraphics[#1]{#2}
      \caption{#3}
  \end{figure}
}

% for cite: #1: author; #2: conference #3: year
\newcommand{\citeinbeamer}[3]{{\tiny{\textcolor{blue}{[#1@#2'#3]}}}}

\newcommand{\term}[1]{\textcolor{blue}{\scriptsize (#1)}}
\newcommand{\set}[1]{\{#1\}}
\newcommand{\question}[1]{\textcolor{red}{\centerline{#1}}}
\newcommand{\answer}[1]{\textcolor{blue}{\centerline {#1}}}
\newcommand{\alertred}[1]{\textcolor{red}{#1}}
\newcommand{\alertblue}[1]{\textcolor{blue}{#1}}
\newcommand{\todo}[1]{\textcolor{red}{\textbf{TODO:} #1}}
\newcommand{\mathbfblue}[1]{\textcolor{blue}{$\mathbf{#1}$}}

\newcommand{\papertitle}{分布数据一致性技术研究}
\newcommand{\idea}{``多样化, 可调节; 精细化, 可度量''}
\newcommand{\ideadt}{``多样化, 可调节''}
\newcommand{\idearm}{``精细化, 可度量''}
%%%%%%%%%%%%%%%%%%%%%%%%%%%%%%%%%%%%%%%%%%%%%%%%%%%%%%%%%%%%%%%%%%%%%%%%%%%%%%%%	
\title[\papertitle]{\papertitle}
\subtitle{(博士学位论文答辩报告)}

\author[魏恒峰]{答辩人: 魏恒峰\\导师: 吕建教授、黄宇副教授}
\institute{南京大学软件所}
\date{\zhtoday}

\AtBeginSection[]{
  \begin{frame}[noframenumbering, plain]
    \frametitle{\papertitle}
    \tableofcontents[currentsection, sectionstyle=show/shaded, subsectionstyle=hide/hide/hide]
  \end{frame}
}

% Delete this, if you do not want the table of contents to pop up 
% at the beginning of each subsection:
% \AtBeginSubsection[]{
%   \begin{frame}[noframenumbering, plain]
%     \frametitle{\papertitle}
%       \tableofcontents[currentsection, sectionstyle=show/shaded, subsectionstyle=show/shaded/hide]
%   \end{frame}
% }

% \bibliography{phd-defense-report.bib}
%%%%%%%%%%%%%%%%%%%%
\begin{document}

\renewcommand\figurename{图} 
\renewcommand\tablename{表} 

% mdf: mdframed; #1: frame color; #2: frame title color; #3: frame title; #4: text color; #5: text
\newcommand{\mdf}[5]{
\begin{mdframed}[frametitle={
  \tikz[baseline = (current bounding box.east), outer sep = 0pt]
  \node[anchor = east, rectangle, fill = #1!20, font = \small]{\strut \textcolor{#2}{#3:}};},
  innertopmargin = 2pt, linecolor = #1!20, linewidth = 2pt, topline = true,
  frametitleaboveskip=\dimexpr-\ht\strutbox\relax]
  \begin{center}
	\textcolor{#4}{#5}
  \end{center}
\end{mdframed}
}

\maketitle

\begin{frame}[noframenumbering, plain]
  \frametitle{\papertitle}
  \tableofcontents[currentsection, sectionstyle=show, subsectionstyle=show/show/hide]
\end{frame}

% \begin{frame}[noframenumbering, plain]
%   \frametitle{分布数据一致性理论与技术研究}
%   \begin{enumerate}
%     \setlength{\itemsep}{15pt}
% 	\item {\bf 研究背景:} {\footnotesize 分布数据}
% 	\item \textcolor{blue}{研究问题:} {\footnotesize 数据一致性}
% 	\item \textcolor{blue}{研究方法:} {\footnotesize 理论模型 + 技术框架}
% 	\item \textcolor{blue}{主要工作:} {\footnotesize VPC + PA2AM + RVSI}
% 	\item \textcolor{blue}{未来工作}
%   \end{enumerate}
% \end{frame}

% \section{研究背景}

%%%%%%%%%%%%%%%
\begin{frame}{分布式应用}
  \fignocaption{width = 0.45\textwidth}{figures/sina-weibo-world-map.pdf}

  \begin{columns}
	\column{0.50\textwidth}
	新浪微博社交网站 \footnotemark:
	\begin{itemize}
	  \item 日均用户近一亿
	  \item 日均消息近一亿条
	\end{itemize}

	\column{0.50\textwidth}
	底层数据服务系统特性需求 \term{$H^{3}L$}: 
	\begin{itemize}
	  \item 低延迟, 高可用性 (4个9 \footnotemark)
	  \item 高容错性, 高可扩展性
	\end{itemize}
  \end{columns}
  
  \footnotetext[1]{2015第三季度; 数据来自 \href{https://www.chinainternetwatch.com/15740/weibo-q3-2015/}{China Internet Watch}.}
  \footnotetext[2]{数据来自 \href{http://www.infoq.com/cn/articles/weibo-platform-availability-9999}{InfoQ}.}
\end{frame}
%%%%%%%%%%%%%%%
\begin{frame}{分布数据}
  \graphicspath{{tikz-in-beamer/}}
  \begin{figure}[h!]
    \centering
    \begin{adjustbox}{max totalsize = {0.60\textwidth}{1.00\textheight}, center}
	  \input{tikz-in-beamer/distributed-data-overlay-for-background}
    \end{adjustbox}
  \end{figure}

  \only<1-3>{应用数据:} \only<4>{\textcolor{red}{分布数据 \term{distributed data}:}}
  \begin{enumerate}
	\item<2-> 分区 \term{partition}: 水平扩展
	\item<3-> 副本 \term{replication}: 就近访问, 容灾备份
  \end{enumerate}
\end{frame}
%%%%%%%%%%%%%%%
\begin{comment}
\begin{frame}{分布共享数据服务}
  \fignocaption{width = 0.55\textwidth}{figures/sina-weibo-redis.pdf}

  \begin{center}
	分布共享数据服务 \term{中间件}

	\term{Distributed Shared Data Service}
	\vspace{0.30cm}

	屏蔽底层数据分布性\; 提供共享数据抽象\; 简化上层应用开发
  \end{center}
\end{frame}
%%%%%%%%%%%%%%%
\begin{frame}{分布共享数据服务典型应用 (I)}
  \fig{width = 0.55\textwidth}{figures/dsss.pdf}
  {分布式存储系统 (\textcolor{blue}{\scriptsize 开源 [左]} \& \textcolor{red}{\scriptsize 商用 [右]}).}

  % \begin{description}
  %   \item[低延迟:] 就近访问副本数据
  %   \item[高可用性, 高容错性:] 备份容灾 
  % \end{description}
\end{frame}
%%%%%%%%%%%%%%%
\begin{frame}{分布共享数据服务典型应用 (II)}
  \fig{width = 0.75\textwidth}{figures/file-share.pdf}
  {个人多设备文件共享 {(\textcolor{blue}{\scriptsize [基于云] C/S 结构 [左]} \& 
  \textcolor{red}{\scriptsize P2P 结构 [右]}).}}

  \begin{description}
    \item[功能需求:] 文件副本 \citeinbeamer{Strauss}{MIT Thesis}{10}
    \item[网络断连:] 备份容灾; 离线可用
  \end{description}
\end{frame}
\end{comment}
%%%%%%%%%%%%%%%
% \begin{frame}{分布副本数据的典型应用 (三)}
%   \fig{width = 0.80\textwidth}{figures/coordination.pdf}
%   {分布式协同应用(上)及服务(下).}
% 
%   分布式协同服务需求 \citeinbeamer{Burrows}{OSDI}{06} vs. ``副本技术'':
%   \begin{description}
%     \item[高性能:] 低延迟就近``读''副本 \citeinbeamer{Yahoo!}{USENIXATC}{10}
%     \item[高可靠性:] 避免单点协同故障
%   \end{description}
% \end{frame}
%%%%%%%%%%%%%%%

% \section{研究问题}

%%%%%%%%%%%%%%%
\begin{frame}{面向共享数据的编程模型}
  \begin{columns}
    \column{0.50\textwidth}
      \fig{width = 0.60\textwidth}{figures/shared-data.png}{(集中式)共享数据系统 \todo{重绘}.}
    \column{0.50\textwidth}
      \textcolor{red}{共享变量}编程模型 (\todo{happy programmer}): 
      \begin{itemize}
	\item 数据层: 数据建模为变量
	\item 业务层: 读---计算---写
	\item 变量全局唯一 (one-copy)
	\item \textcolor{cyan}{\footnotesize 隐含假设:} 变量随时可访问
	\item 读操作返回最新值 
      \end{itemize}
  \end{columns}
\end{frame}
%%%%%%%%%%%%%%%
\begin{frame}{面向分布数据的编程模型 (I)}
  \begin{columns}
    \column{0.50\textwidth}
      \fig{width = 0.80\textwidth}{figures/distributed-data.pdf}{分布数据系统.}
    \column{0.50\textwidth}
      共享变量编程模型与分布数据不匹配:
      \begin{itemize}
	\item 数据副本 (replication)
	\item 没有全局唯一变量的概念
	\item 节点/通讯故障
	\item 读操作语义无定义
      \end{itemize}
  \end{columns}
\end{frame}
%%%%%%%%%%%%%%%
\begin{frame}{面向分布数据的编程模型 (II)}
  \begin{columns}[t]
    \column{0.50\textwidth}
      \textcolor{red}{消息传递}编程模型:
      \begin{itemize}
	\item 读/写 + 通信 (communication)
	\item 从哪里读
	\item 写哪里去
	\item 如何理解返回值
	\item 如何处理失败
      \end{itemize}
    \column{0.50\textwidth}
      \pause
      消息传递编程模型的缺点: (\todo{unhappy programmer})
      \begin{description}
	\item[数据层:] 暴露分布数据细节
	\item[业务层:] 难于编程
	\item[数据层:] 数据语义不明确
	\item[业务层:] 难于保证程序正确性
      \end{description}
  \end{columns}
\end{frame}
%%%%%%%%%%%%%%%
\begin{frame}{面向分布数据的编程模型 (III)}
  \begin{center}
    共享变量编程模型与分布数据不匹配的根本原因: 

    \textcolor{blue}{\large 分布数据一致性问题}
  \end{center}

  \begin{cdef}[数据一致性问题 (非形式化定义)]
    读操作语义问题: 在分布数据环境下, 读操作允许返回什么值?
  \end{cdef}

  \pause
  \begin{alertblock}{如何理解数据一致性问题:}
    \begin{enumerate}
      \item 数据层数据一致与否是相对于业务层逻辑而言的
      \item 可通过业务层出现的异常行为(anomalies)来衡量数据是否一致
    \end{enumerate}
  \end{alertblock}
\end{frame}
%%%%%%%%%%%%%%%
\begin{frame}{数据一致性问题举例 (I)}
  \fig{width = 0.60\textwidth}{figures/data-inconsistency-comment-reordering.pdf}
  {社交网络中, 消息-评论乱序 \citeinbeamer{Lloyd}{CACM}{14}.}
\end{frame}
%%%%%%%%%%%%%%%
\begin{frame}{数据一致性问题举例 (II)}
  \begin{figure}[h!]
    \centering
    \begin{adjustbox}{max totalsize = {0.65\textwidth}{0.65\textheight}, center}
      %        File: data-inconsistency-rmw.tex
%     Created: Thu Oct 08 10:00 AM 2015 C
% Last Change: Thu Oct 08 10:00 AM 2015 C
%
\begin{tikzpicture}
  \uncover<1->{
  \begin{scope}
    % phone
    \node (phone) [] at (0,0) {\includegraphics[scale = 0.4]{figures/phone-icon.png}};
    % tablet
    \node (tablet) [below left = 4.0cm and 3.0cm of phone] {\includegraphics[scale = 
    0.55]{figures/tablet-icon.jpg}};
    % pc
    \node (pc) [below right = 4.0cm and 3.0cm of phone] {\includegraphics[scale = 
    0.60]{figures/pc-icon}};
  \end{scope}
  }

    % write from pc 
  \uncover<2->{
    \begin{scope}[<->, blue, line width = 6pt, font = \huge]
      % person at pc
      \only<2>{
      \node (person-pc) [above left = -1.0cm and -2.0cm of pc] {\includegraphics[scale = 
      0.30]{figures/person-icon}};
      }
      \draw [] (pc.west) to node [midway, sloped, above] {\textbf{1.} update file $f$} 
      (tablet.east);
      \draw [red, loosely dashed] (pc.north) to [out = 90, in = -20] node [midway, sloped, font = 
      \Huge, scale = 2]{$\times$} (phone.east);
      \draw (pc) to [loop, out = 60, in = -10, looseness = 3] node [midway, above, sloped] 
      {\textbf{1.} update file $f$} (pc);
    \end{scope}
   }


    % read from phone
   \uncover<3->{
    \begin{scope}[<->, brown, line width = 6pt, font = \huge]

      % person at phone
      \node (person-phone) [above left = -1.0cm and -2.0cm of phone] {\includegraphics[scale = 
      0.30]{figures/person-icon}};

      \draw (phone) to [loop, out = 180, in = -100, looseness = 3] node [midway, below, sloped] 
      {\textbf{2.} read file $f$} (phone);   
    \end{scope}

    % update lost
    \node (inconsistency) [font = \huge, right = of person-phone, red] {\bf Update Lost!};
  }
\end{tikzpicture}

    \end{adjustbox}
    \caption{多设备文件共享时, 更新丢失 ($\#N = 3, \#W = 2, \#R = 1$).}
  \end{figure}
\end{frame}
%%%%%%%%%%%%%%%
\begin{frame}{面向分布数据的编程模型 (IV)}
  \begin{center}
    共享变量编程模型与分布数据不匹配: 数据一致性问题 
  \end{center}

  \vspace{0.50cm}
  消息传递编程模型: 业务层显式处理数据一致性问题
  \begin{itemize}
    \item 理论难度大, 实现繁琐
    \item 处理不当: 异常
  \end{itemize}
\end{frame}
%%%%%%%%%%%%%%%
\begin{frame}{面向分布数据的编程模型 (V)}
  \begin{cquestion}
    如何解决共享变量编程模型与分布数据不匹配的问题?
  \end{cquestion}

  \vspace{0.30cm}
  \begin{center}
    {\large \textcolor{red}{分布共享数据}服务: 在\textcolor{blue}{分布数据}之上提供\textcolor{blue}{共享数据}的假象}
  \end{center}
  
  \begin{description}
    \item<2->[业务层:] 采用``自然的''共享变量编程模型
    \item<3->[分布共享数据服务:] 专注分布数据一致性问题, 实现分布数据透明性
      \begin{itemize}
	\item 屏蔽消息传递编程模型的通信细节
	\item 形式化定义读操作语义
      \end{itemize}
    \item<2->[数据层:] 分布数据
  \end{description}
\end{frame}
%%%%%%%%%%%%%%%
\begin{frame}[fragile]{分布共享数据服务应用举例 (I)}
   \lstset{language=C++,
           basicstyle=\ttfamily\scriptsize,
           keywordstyle=\color{blue}\ttfamily,
           stringstyle=\color{red}\ttfamily,
           commentstyle=\color{green}\ttfamily,
	   breaklines=true,
	   captionpos=b,
	   caption=在社交网络应用中使用满足``因果一致性''的共享变量.
          }
    \begin{lstlisting}
      // Alice's post
      put(key = lost_msg, val = "lost")
      put_after(key = found_msg, val = "found", dep = lost_msg)

      // Bob's reply
      get(key = found_msg)  // get "found"
      put(key = glad_msg, val = "glad", dep = found_msg)

      // Eve's view
      read(key = glad_msg)  // get "glad"
      read(key = found_msg) // get "found" instead of "NULL"
    \end{lstlisting}
\end{frame}
%%%%%%%%%%%%%%%
\begin{frame}{分布共享数据服务应用举例 (II)}
  Another example: coordination???
\end{frame}
%%%%%%%%%%%%%%%

% \section{研究方法}

%%%%%%%%%%%%%%%%%%%%%%%%%%%%%%
\subsection{理论模型:分布共享数据}

%%%%%%%%%%%%%%%
\begin{frame}{分布共享数据 (I)}
  \fig{width = 0.40\textwidth}{figures/dsm4replicas-model.pdf}{分布数据共享服务.}
\end{frame}
%%%%%%%%%%%%%%%
\begin{frame}{分布共享数据 (II)}
  \begin{center}
    \textcolor{cyan}{$x, y:$} 共享变量 \hspace{0.30cm} \textcolor{cyan}{$p_0, p_1:$} 
    客户进程
  \end{center}

  多进程并发提交 (读/写) 操作:
  
  \fignocaption{width = 0.55\textwidth}{figures/register-what-value.pdf}

  \begin{center}
    \question{问题: 读操作允许返回什么值?}
  \end{center}

  \answer{
  \[
    \text{不同一致性}
    \xrightleftharpoons[\text{定义}]{\,\text{规定}\,}
    \text{不同合法返回值}
  \]} 
\end{frame}
%%%%%%%%%%%%%%%
\begin{frame}{分布共享数据 (III)}
  \begin{center}
    \textcolor{blue}{\large 基本定位: 传统概念应用于新型平台}
  \end{center}

  \vspace{0.20cm}
  \begin{center}
    分布共享数据服务: 分布共享内存模型 + 分布数据系统
  \end{center}
\end{frame}
%%%%%%%%%%%%%%%
%%%%%%%%%%%%%%%%%%%%%%%%%%%%%%
\subsection{技术途径: 三维框架}

%%%%%%%%%%%%%%%
\begin{frame}{分布共享内存中的数据一致性问题}
      数据一致性问题的三个层面: 
      \begin{description}
	\item[1. 虚拟共享数据有什么?]
	  \begin{itemize}
	    \item 数据类型
	  \end{itemize}
	\item[2. 上层接口语义是什么?]
	  \begin{itemize}
	    \item 一致性模型
	  \end{itemize}
	\item[3. 底层消息传递为什么?]
	  \begin{itemize}
	    \item 一致性保障
	  \end{itemize}
      \end{description}
\end{frame}
%%%%%%%%%%%%%%%
\begin{frame}{研究框架}
  \fig{width = 0.75\textwidth}{figures/thesis-proposal-3d-framework-allinone.pdf}{数据一致性及保障技术研究框架}
\end{frame}
%%%%%%%%%%%%%%%

% \section{相关工作}	\label{section:related-work}

%%%%%%%%%%%%%%%
\begin{frame}{相关工作分类}
  \begin{table}[]
	\centering
	\caption{\idea{}研究理念相关工作.}
	\label{tbl:related-work-categories}
	\renewcommand\arraystretch{2}
	\resizebox{\textwidth}{!}{%
	  \begin{tabular}{cc|c|c|c|c|}
		\cline{3-6}
		\multicolumn{2}{c|}{} & \multicolumn{2}{c|}{\textbf{读写寄存器}} & \multicolumn{2}{c|}{\textbf{事务}} \\ \cline{3-6}
		\multicolumn{2}{c|}{} & 多处理器系统 & 分布式系统 & 多处理器系统 & 分布式系统 \\ \hline
		\multicolumn{2}{|c|}{\textbf{\ideadt{}}} &  &  &  &  \\ \hline
		\multicolumn{1}{|c|}{\multirow{2}{*}{\textbf{\idearm{}}}} & 验证 &  &  &  &  \\ \cline{2-6} 
		\multicolumn{1}{|c|}{} & 量化 &  &  &  &  \\ \hline
	  \end{tabular}
    }
  \end{table}
\end{frame}
%%%%%%%%%%%%%%%
\begin{frame}{\ideadt{}的研究理念 (一)}
  \begin{table}[]
	\centering
	\renewcommand\arraystretch{1.5}
	\resizebox{\textwidth}{!}{%
	  \begin{tabular}{cc|c|c|c|c|}
		\cline{3-6}
		\multicolumn{2}{c|}{} & \multicolumn{2}{c|}{\textbf{读写寄存器}} & \multicolumn{2}{c|}{\textbf{事务}} \\ \cline{3-6} 
		\multicolumn{2}{c|}{} & 多处理器系统 & 分布式系统 & 多处理器系统 & 分布式系统 \\ \hline
		\multicolumn{2}{|c|}{\textbf{\ideadt{}}} & \only<1-3>{\textcolor{red}{\bf \checkmark}} 
		\only<4->{\begin{tabular}[c]{@{}c@{}}\textcolor{red}{相关工作丰富}\\ \textcolor{red}{理论扎实}\end{tabular}} &  &  &  \\ \hline
	  \end{tabular}%
	}
  \end{table}

  \pause 

  \begin{description}
	\setlength{\itemsep}{5pt}
	\item[典型:] Hybrid consistency \citeinbeamer{Attiya}{SIAM J. Comput.}{98}
	\item[思想:] 将操作分为强弱两类
	  \pause
	\item[其它:] ``带同步的''一致性模型 \citeinbeamer{Dubois}{IEEE Computer}{88} \citeinbeamer{Steinke}{JACM}{04}
	\item[特点:] 强调正确性 {\footnotesize (properly synchronized)}
  \end{description}
\end{frame}
%%%%%%%%%%%%%%%
\begin{frame}{\ideadt{}的研究理念 (二)}
  \begin{table}[]
	\centering
	\renewcommand\arraystretch{1.5}
	\resizebox{\textwidth}{!}{%
	  \begin{tabular}{cc|c|c|c|c|}
		\cline{3-6}
		\multicolumn{2}{c|}{} & \multicolumn{2}{c|}{\textbf{读写寄存器}} & \multicolumn{2}{c|}{\textbf{事务}} \\ \cline{3-6} 
		\multicolumn{2}{c|}{} & 多处理器系统 & 分布式系统 & 多处理器系统 & 分布式系统 \\ \hline
		\multicolumn{2}{|c|}{\textbf{\ideadt{}}} & {\begin{tabular}[c]{@{}c@{}}{\small 相关工作丰富}\\ {\small 理论扎实}\end{tabular}}
		& \only<1-3>{\textcolor{red}{\bf \checkmark}} 
		\only<4->{\begin{tabular}[c]{@{}c@{}}\textcolor{red}{渐成趋势}\\ \textcolor{red}{理论欠缺}\end{tabular}} &  &  \\ \hline
	  \end{tabular}%
	}
  \end{table}

  \pause 

  \begin{description}
	\setlength{\itemsep}{5pt}
	\item[思想:] 借鉴并发展 Hybrid consistency 的思想
	\item[典型:] 
	  \begin{itemize}
		\item Causal+forced+immediate operations \citeinbeamer{Ladin}{TOCS}{92}
		\item RedBlue consistency \citeinbeamer{Li}{OSDI}{12}
		% \item Apache Cassandra~\footnote{\url{http://cassandra.apache.org/}} \citeinbeamer{Facebook}{SIGOPS OSR}{10}
		\item Pileus \citeinbeamer{Terry}{SOSP}{13}
	  \end{itemize}
	  \pause
	\item[特点:] 更细粒度的多一致性模型共存、更能容忍数据不一致
  \end{description}
\end{frame}
%%%%%%%%%%%%%%%
\begin{frame}{\ideadt{}的研究理念 (三)}
  \begin{table}[]
	\centering
	\renewcommand\arraystretch{1.5}
	\resizebox{\textwidth}{!}{%
	  \begin{tabular}{cc|c|c|c|c|}
		\cline{3-6}
		\multicolumn{2}{c|}{} & \multicolumn{2}{c|}{\textbf{读写寄存器}} & \multicolumn{2}{c|}{\textbf{事务}} \\ \cline{3-6} 
		\multicolumn{2}{c|}{} & 多处理器系统 & 分布式系统 & 多处理器系统 & 分布式系统 \\ \hline
		\multicolumn{2}{|c|}{\textbf{\ideadt{}}} & {\begin{tabular}[c]{@{}c@{}}{\small 相关工作丰富}\\ {\small 理论扎实}\end{tabular}}
		& \begin{tabular}[c]{@{}c@{}}{\small 渐成趋势}\\ {\small 理论欠缺}\end{tabular} 
		& \only<1>{\textcolor{red}{\small 软件事务内存}}\only<2->{\textcolor{gray}{\small 软件事务内存}}
		& \only<2-3>{\textcolor{red}{\bf \checkmark}} 
		  \only<4->{\textcolor{red}{探索阶段}} \\ \hline
		  % \only<4->{\begin{tabular}[c]{@{}c@{}}\textcolor{red}{探索阶段}\\ \textcolor{red}{系统、理论欠缺}\end{tabular}} \\ \hline
	  \end{tabular}
	}
  \end{table}

  \pause

  \begin{description}
	\setlength{\itemsep}{5pt}
	\item[思想:] 多个事务一致性模型共存
	\item[典型:] 
	  \begin{itemize}
		\item RC-SR {\scriptsize (relaxed currency serializability)} \citeinbeamer{Bernstein}{SIGMOD}{06}
		\item Pileus consistency choices \citeinbeamer{Terry}{MSR-TR}{13}
		\item Multi-level CSI {\scriptsize (Causal Snapshot Isolation)} \citeinbeamer{Tripathi}{BigData}{15}
	  \end{itemize}
	  \pause
	\item[挑战:] ``多样化''事务语义; 可扩展的系统实现
  \end{description}
\end{frame}
%%%%%%%%%%%%%%%
\begin{frame}{\idearm{}的研究理念 (一)}
  \begin{table}
	\centering
	\begin{tabular}{|c|c|c|c|c|}
	  \hline
	  & 读写寄存器 & 事务
	\end{tabular}
	\caption{相关工作二: \idearm{}的研究理念}
	\label{tbl:idearm-related-work}
  \end{table}
\end{frame}
%%%%%%%%%%%%%%%
%%%%%%%%%%%%%%%

\section{本文工作}

%%%%%%%%%%%%%%%%%%%%%%%%%%%%%%%%%%%%%%%%	
\subsection{概述}

%%%%%%%%%%%%%%%
\begin{frame}{工作概述}
  \begin{table}[]
	\centering
	\caption{本文落实\idea{}研究理念.}
	\renewcommand\arraystretch{1.6}
	\resizebox{\textwidth}{!}{%
	  \begin{tabular}{cc|c|c|c|c|}
		\cline{3-6}
		\multicolumn{2}{c|}{} & \multicolumn{2}{c|}{\textbf{读写寄存器}} & \multicolumn{2}{c|}{\textbf{事务}} \\ \cline{3-6} 
		\multicolumn{2}{c|}{} & 多处理器系统 & \cellcolor{red!80}{分布式系统} & \textcolor{gray}{多处理器系统} & \cellcolor{red!80}{分布式系统} \\ \hline

		\multicolumn{2}{|c|}{\textbf{\ideadt{}}} & {\begin{tabular}[c]{@{}c@{}}{\small 相关工作丰富}\\ {\small 理论扎实}\end{tabular}}
		& \begin{tabular}[c]{@{}c@{}}{\small 渐成趋势}\\ {\small 理论欠缺}\end{tabular} 
		& \multirow{2}{*}[-3em]{\begin{tabular}[c]{@{}c@{}}\textcolor{gray}{\small 软件}\\ \textcolor{gray}{\small 事务内存}\end{tabular}}
		& \cellcolor{brown!80}{\begin{tabular}[c]{@{}c@{}}{\small 探索阶段}\\ {(RVSI)}\end{tabular}} \\ \cline{1-4} \cline{6-6}

		\multicolumn{1}{|c|}{\multirow{2}{*}[-1em]{\textbf{\idearm{}}}} & 验证 
		& \begin{tabular}[c]{@{}c@{}}{\small 典型模型}\\ {\small 理论全面}\end{tabular} 
		& \cellcolor{brown!80}{\begin{tabular}[c]{@{}c@{}}{\small 弱模型验证}\\ {(VPC)}\end{tabular}}
		& 
		& \begin{tabular}[c]{@{}c@{}}{\small 理论全面}\\ {\small 指导协议设计}\end{tabular}
		\\ \cline{2-4} \cline{6-6} \hhline{*{3}{~}-*{2}{~}}

		\multicolumn{1}{|c|}{} & 量化 
		& \begin{tabular}[c]{@{}c@{}}{\small 暂无}\\ {\small 强调正确性}\end{tabular}
		& \cellcolor{brown!80}{\begin{tabular}[c]{@{}c@{}}{\small 量化协议}\\ {(PA2AM)}\end{tabular}}
		&  
		& \begin{tabular}[c]{@{}c@{}}{\small 量化协议难}\\ {\small 相关工作少}\end{tabular} 
		\\ \hline
	  \end{tabular}
	}
  \end{table}
\end{frame}
%%%%%%%%%%%%%%%
%%%%%%%%%%%%%%%%%%%%%%%%%%%%%%%%%%%%%%%%
\subsection{VPC: Pipelined-RAM 一致性验证}

\newcommand{\pram}{Pipelined-RAM}
\newcommand{\vpc}[1]{\ifthenelse{\isempty{#1}{}}{\textsf{VPC}}{\textsf{VPC-\MakeUppercase{#1}}}} 
\newcommand{\npc}{$\sf{NP}$-complete}
\newcommand{\npcn}{$\sf{NP}$-completeness}
\newcommand{\rwclosure}{\textsc{RW-Closure}}
\newcommand{\readcentric}{\textsc{Read-Centric}}
%%%%%%%%%%%%%%%
\begin{frame}{VPC 工作在技术框架中的位置}
  \fig{width = 0.50\textwidth}{figures/3d-framework-vpc.pdf}{VPC --- \pram{} 一致性验证.}
\end{frame}
%%%%%%%%%%%%%%%
\begin{frame}{VPC 问题定义}
  \begin{cdef}[VPC: Verifying PRAM Consistency]
    VPC 判定问题:
	\vspace{8pt}
    \begin{description}
	  \setlength{\itemsep}{8pt}
      \item[实例:] 系统执行 {\small (execution $e$; 即, 读写操作序列)}
      \item[问题:] 该执行是否满足 PRAM 一致性模型 {\small ($\mathcal{C}$)}? 
		\[
		  e \in \mathcal{C} \Rightarrow \set{0,1}?
		\]
    \end{description}    
  \end{cdef}

  \vspace{0.30cm}
  实例规模 $n$: 系统执行中操作的总数
\end{frame}
%%%%%%%%%%%%%%%
\begin{frame}{研究动机}
  \question{问题: 为什么要验证 Pipelined-RAM {\small (PRAM)} 一致性?}
  \vspace{0.10cm}

  \begin{description}
    \setlength{\itemsep}{10pt}
    \item[验证:] 用户与商家就数据一致性签订 SLA 协议 \\
	  \citeinbeamer{Amazon}{SOSP}{07} \citeinbeamer{Golab}{PODC}{11}
	  \vspace{2pt}
	  \begin{itemize}
		\item \textcolor{teal}{[商家]} 系统测试手段之一
		\item \textcolor{teal}{[用户]} 确认系统是否提供了其所声称的数据一致性 
	  \end{itemize}
	\pause
	\item[PRAM:] 存储系统常提供``会话'' {\small (session)} 一致性\\
      \citeinbeamer{Saito}{CSUR}{05} \citeinbeamer{Terry}{CACM}{13}
	  \vspace{2pt}
      \begin{itemize}
		\item 包含了弱一致性的诸多变体 
		\item 近似于 PRAM 一致性 \citeinbeamer{Brzezi$\acute{\text{n}}$ski}{PDP}{04} \citeinbeamer{Bailis}{VLDB}{13}
      \end{itemize}
  \end{description}
\end{frame}
%%%%%%%%%%%%%%%
\begin{frame}{VPC 问题分类}
  \begin{table}[!t]
    \centering
	\caption{VPC 问题的四种变体 (按``执行''的类型) 及复杂度分析 ($\textcolor{red}{[\ast]}$\textcolor{red}{: 本文工作}).}
	\renewcommand\arraystretch{1.2}
    \begin{tabular}{|c|c|c|}
      \hline
      & \it (S)ingle variable  & \it (M)ultiple variables
      \\ \hline
	  {\it write (D)uplicate values} &
	  \innercell{c}{VPC-SD \\ (\npc{}) $\textcolor{red}{[\ast]}$} &
	  \innercell{c}{VPC-MD \\ (\npc{}) $\textcolor{red}{[\ast]}$}
      \\ \hline
	  \only<1>{\it write (U)nique value}\only<2>{\cellcolor{brown!80}{\it write (U)nique value}} &
      \innercell{c}{VPC-SU \\ (P) \citeinbeamer{Golab}{PODC}{11}} &
      \innercell{c}{VPC-MU \\ (P) $\textcolor{red}{[\ast]}$}
      \\ \hline
    \end{tabular}
  \end{table}

  \vspace{10pt}
  
  \uncover<2->{\textcolor{brown}{\centerline{Read-mapping \citeinbeamer{Gibbons}{SICOMP}{97}: $\forall r, f(r) = w$.}}}
\end{frame}
%%%%%%%%%%%%%%%
\begin{frame}{VPC-SD (VPC-MD) 是 \npc{} 问题}
  \fig{width = 0.40\textwidth}{figures/vpcsd-npc.pdf}{对应于 \textsc{Unary 3-Partition} 实例 $A = \{2,2,1,1,1,1\}, m = 2, B = 4$ 的 VPC 执行.} 
\end{frame}
%%%%%%%%%%%%%%%
\begin{frame}{VPC-MU 的多项式算法 \rwclosure{}}
  \begin{figure}[h!]
    \centering
    \begin{adjustbox}{max totalsize = {0.65\textwidth}{1.00\textheight}, center}
	  \input{tikz-in-beamer/rw-closure-example}
    \end{adjustbox}
	\caption{\rwclosure{} 算法示例: 在传递闭包之上迭代应用 $w'wr$ 规则.}
  \end{figure}

  TODO: 规则c
\end{frame}
%%%%%%%%%%%%%%%
\begin{frame}{VPC-MU 的多项式算法 \rwclosure{}}
  \begin{ctheorem}[\rwclosure{} 算法正确性]
	\vpc{mu} 实例满足 \pram{} 一致性当且仅当 \rwclosure{} 算法所得图是有向无环图.
  \end{ctheorem}

  \pause 
  TODO: proof

  \pause
  \centerline{\rwclosure{} 算法复杂度: $O(n^2) \cdot O(n^3) = O(n^5)$}
\end{frame}
%%%%%%%%%%%%%%%
\begin{frame}{VPC-MU 的多项式算法 \readcentric{}}
  关键点:

  \begin{ctheorem}[\rwclosure{} 算法正确性]
	\vpc{mu} 实例满足 \pram{} 一致性当且仅当 \readcentric{} 算法所得图是有向无环图.
  \end{ctheorem}

  \pause
  \centerline{\rwclosure{} 算法复杂度: }
\end{frame}
%%%%%%%%%%%%%%%
\begin{frame}{实验评估}
  
\end{frame}
%%%%%%%%%%%%%%%
\begin{frame}{VPC 的意义}
  \mdf{red}{blue}{}{teal}{
	\begin{enumerate}
	  \item \readcentric{} 算法可用于测试系统是否正确实现了 \pram{} 一致性模型
	  \item \npcn{} 结果有助于理解弱一致性模型的复杂度
	\end{enumerate}
  }
\end{frame}
%%%%%%%%%%%%%%%

%%%%%%%%%%%%%%%%%%%%%%%%%%%%%%%%%%%%%%%%	
\subsection{PA2AM: (2-)Atomicity 一致性维护与量化}

%%%%%%%%%%%%%%%
\begin{frame}{PA2AM 工作在技术框架中的位置}
  \fig{width = 0.50\textwidth}{figures/3d-framework-pa2am.pdf}{PA2AM --- (2-)Atomicity 一致性维护与量化.}
\end{frame}
%%%%%%%%%%%%%%%
\begin{frame}{研究动机}
  \question{问题: 为什么提出 probabilistically-atomic 2-atomicity {\small (PA2AM)} 一致性?}
  \vspace{0.30cm}

  \pause
  ``数据一致性/访问延迟'' PACELC 权衡 \citeinbeamer{Abadi}{IEEE Computer}{12}:

  \fignocaption{width = 0.35\textwidth}{figures/stronger-consistency-tradeoff.pdf}

  % \begin{quote}
  %   ``As soon as a distributed storage system replicates data, a \textcolor{brown}{tradeoff 
  %   between consistency and latency} arises.''
  % \end{quote}

  \pause

  ``低延迟''至关重要:
  \begin{itemize}
	\item 100ms 额外延迟 $\Rightarrow$ 1\% 销售下滑 \citeinbeamer{Amazon}{Blog}{06}
	\item 100$\sim$400ms 额外延迟 $\Rightarrow$ 0.2\%$\sim$0.6\% 搜索量下降 \citeinbeamer{Google}{Blog}{09}
  \end{itemize}

  % {\small
  % \begin{table}
  %   \begin{tabular}{c|c}
  %     \textcolor{blue}{\bf 系统} & \textcolor{blue}{\bf 一致性}		\\ \hline
  %     Dynamo@Amazon \citeinbeamer{Amazon}{SOSP}{07} & eventual consistency \\ \hline
  %     PNUTS@Yahoo! \citeinbeamer{Yahoo!}{PVLDB}{08} & cache consistency \\ \hline
  %     Tao@Facebook \citeinbeamer{Facebook}{ATC}{13} & $\le$ read-after-write \\ \hline
  %   \end{tabular}
  % \end{table}
  % }
\end{frame}
%%%%%%%%%%%%%%%
\begin{frame}{PA2AM 一致性}
  % \begin{cdef}[``近乎强''一致性]
  %   对某特定强一致性的弱化: \begin{itemize}
  %     \item (版本) 允许读陈旧值,但陈旧度有限
  %     \item (概率) 读到陈旧值的概率很小
  %   \end{itemize}
  % \end{cdef}

  \begin{center}
	\only<2->{\textcolor{blue}{``近乎强''一致性: }}{在保证\textcolor{red}{低延迟}的情况下获得\textcolor{red}{尽可能强}的数据一致性.}
  \end{center}

  \uncover<3->{
  \begin{cdef}[PA2AM 一致性]
	\begin{description}
	  \setlength{\itemsep}{5pt}
	  \item[低延迟:] \uncover<4->{读操作只需一轮网络通信} 
	  \item[尽可能强:] \uncover<6->{对 atomicity {\small (strongest)} 的弱化}
    \begin{itemize}
	  \item<6-> \textcolor{brown}{\it (版本)} 2-atomicity: 允许读陈旧值, 但\textcolor{red}{陈旧度 $k \le 2$}
	  \item<6-> \textcolor{brown}{\it (概率)} \textcolor{red}{$\mathbb{P}(k = 2)$ 很小}
    \end{itemize}
    \end{description}
  \end{cdef}
  }

  \vspace{0.20cm}
  \uncover<5->{
  \begin{ctheorem}[不可能性结果]
    (单写模型下) 不存在低延迟的 atomicity 维护算法 \citeinbeamer{Dutta}{PODC}{04}.
  \end{ctheorem}
  }
\end{frame}
%%%%%%%%%%%%%%%
\begin{frame}{PA2AM 维护算法}
  \fig{width = 0.80\textwidth}{figures/atomicity-2am-read-compare.pdf}
  {经典 atomicity 算法中, 读操作需两轮网络通信 \citeinbeamer{Attiya}{JACM}{95} 
  \citeinbeamer{Dutta}{PODC}{04}. 
  PA2AM 算法实现 2-atomicity {\scriptsize (单写模型下)}, 读操作只需一轮网络通信.}
\end{frame}
%%%%%%%%%%%%%%%
\begin{frame}{PA2AM 量化分析}
  \question{问题: PA2AM 算法在多大程度上违反了 atomicity?}
  \vspace{0.10cm}

  \begin{itemize}
    \setlength{\itemsep}{10pt}
	\item 充要条件: ONI (old-new inversion) \citeinbeamer{Attiya}{JACM}{95}
      \fignocaption{width = 0.45\textwidth}{figures/2atomicity-case.pdf}
    \item PA2AM 量化分析: 计算 $\mathbb{P}(\textrm{ONI})$, 其值越小越好
      \begin{enumerate}
        \setlength{\itemsep}{3pt}
        \item $\textrm{ONI} \triangleq \textrm{CP} \cap \textrm{RWP}$
        \item 排队论建模, 计算 $\mathbb{P}(\textrm{CP})$ \item 带时间的球盒模型, 计算 
          $\mathbb{P}(\textrm{RWP|CP})$
        % \item 实验统计 $\mathbb{P}(\textrm{CP})$, $\mathbb{P}(\textrm{RWP|CP})$ 与 
        % $\mathbb{P}(\textrm{ONI})$, 以作对照
      \end{enumerate}
  \end{itemize}
\end{frame}
%%%%%%%%%%%%%%%
\begin{frame}{PA2AM 量化分析}
  公式推导:
  \begin{figure}
	\begin{subfigure}{0.50\textwidth}
	  \centering
	  \includegraphics[width = 0.70\textwidth]{figures/cp.png}
	\end{subfigure}%
	\begin{subfigure}{0.50\textwidth}
	  \centering
	  \includegraphics[width = 0.70\textwidth]{figures/rwp.png}
	\end{subfigure}
  \end{figure}

  \vspace{0.10cm}

  数值结果 (左一) 与实验结果 (右二): 
  \begin{figure}
	\begin{subfigure}{0.45\textwidth}
	  \centering
	  \includegraphics[width = 0.60\textwidth]{figures/oni-pgfplot.pdf}
	\end{subfigure}%
	\begin{subfigure}{0.55\textwidth}
	  \centering
	  \includegraphics[width = 0.90\textwidth]{figures/experiment-oni-pgfplot.pdf}
	\end{subfigure}
  \end{figure}
\end{frame}
%%%%%%%%%%%%%%%
\begin{frame}{PA2AM vs. 弱一致性模型}
  \begin{figure}
	\begin{subfigure}{0.50\textwidth}
	  \centering
	  \includegraphics[width = 0.85\textwidth]{figures/rwn-2am-read-latency-quantiles.pdf}
	  \caption{读操作延迟对比.}
	\end{subfigure}%
	\begin{subfigure}{0.50\textwidth}
	  \centering
	  \includegraphics[width = 0.70\textwidth]{figures/rwn-maj.pdf}
	  \caption{$k$-陈旧度对比.}
	\end{subfigure}
	\caption{PA2AM 与 eventual consistency (RWN-Maj 协议) 对比结果.}
  \end{figure}

  \vspace{-0.50cm}
  \begin{table}[]
  \renewcommand{\arraystretch}{1.3}
  \centering
  \caption{不同 $R,W,N$ 配置下, RWN-All 执行中具有不同陈旧度的读操作的比率.}
  \label{tbl:rwn-all-staleness}
  \resizebox{\textwidth}{!}{%
  \begin{tabular}{|c||c|c|c|c||c|c|c|c|c|c|c|}
  \hline
  {\bfseries \# replicas} & \multicolumn{4}{c||}{\bfseries replica factor = 3 ($400,000$
    \textsl{read} operations)} & \multicolumn{7}{c|}{\bfseries replica factor = 5 ($800,000$
    \textsl{read} operations)} 
  \\ \hline
  {\bfseries $R,W,N$} % \textrm{R}+\textrm{W} $\boldsymbol{\le}$ \textrm{N}}  
  & $1+1<3$	& $1+2=3$	& $2+1=3$	& $2+2>3 \;(\text{PA2AM})$ 
  & $1+2<5$ 	& $1+3<5$ 	& $1+4=5$	& $2+2<5$ 	& $2+3=5$    & $3+2=5$     & $3+3>5 \;(\text{PA2AM})$     
  \\ \hline \hline
  {\boldmath $\max k$}          & $6$      & $4$ 	& $2$    &  {\boldmath $1$}
  & $2$	& $2$      	& $2$	   & $2$	& $2$	 &  $2$     & {\boldmath $1$}
  \\ \hline
  {\bfseries $\sum_{k \ge 1}$-staleness}       & $0.0084125$      & $0.000315$ 	&  $0.0004675$	&
  {\boldmath $0.000085$}
  	&  $0.00377875$	&  $0.002755$	&  $0.00406$     & $0.0027225$      & $0.0020275$     & $0.002255$      
	&  {\boldmath $0.0003525$}
  \\ \hline
  \end{tabular}%
}
\end{table}

\end{frame}
%%%%%%%%%%%%%%%
\begin{frame}{PA2AM 的意义}
  \mdf{red}{blue}{PA2AM 可作为一致性/延迟权衡的一种有价值的选项}{teal}{
	\begin{itemize}
	  \item 既 {\small (在统计意义上)} 满足强一致性模型对数据一致性的高标准
	  \item 又具有弱一致性模型的性能优势
    \end{itemize}
  }
\end{frame}
%%%%%%%%%%%%%%%

%%%%%%%%%%%%%%%%%%%%%%%%%%%%%%%%%%%%%%%%	
\subsection{RVSI: Snapshot Isolation 一致性弱化与维护}

\newcommand{\chameleon}{$\textsc{Chameleon}^{\textsc{\scriptsize TKVS}}$}
\newcommand{\konebv}{$k_1$-BV}
\newcommand{\ktwofv}{$k_2$-FV}
\newcommand{\kthreesv}{$k_3$-SV}
\newcommand{\mpord}[1]{\mathcal{O}_{#1}}
\newcommand{\tsts}[1]{#1.{sts}}
\newcommand{\tcts}[1]{#1.{cts}}
%%%%%%%%%%%%%%%
\begin{frame}{RVSI 工作在技术框架中的位置}
  \fig{width = 0.50\textwidth}{figures/3d-framework-rvsi.pdf}
	{RVSI --- Snapshot Isolation 一致性弱化与维护.}
\end{frame}
%%%%%%%%%%%%%%%
\begin{frame}{研究动机}
  \question{问题: 为什么要提出 Relaxed Version Snapshot Isolation {\small (RVSI)} 一致性?}
  \vspace{0.15cm}

  \begin{description}
    \setlength{\itemsep}{5pt}
    \item[分布式事务:]
      \begin{itemize}
        \item ``all-or-none'' 语义
		\item 受到分布式存储系统的关注 \href{https://issues.apache.org/jira/browse/CASSANDRA-7056}{\citeinbeamer{Cassandra}{CASSANDRA-ISSUE-7056}{14}}
      \end{itemize}
	\item[弱一致性:] \textcolor{red}{SI} \citeinbeamer{Berenson}{SIGMOD}{95} \textcolor{red}{$\xRightarrow{\text{relaxed}}$}\\
        PCSI \citeinbeamer{Elnikety}{SRDS}{05} PSI \citeinbeamer{Sovran}{SOSP}{11} NMSI \citeinbeamer{Ardekani}{SRDS}{13} 
    \pause
	\vspace{0.30cm}
	\item[\textcolor{red}{异常控制:}] 容忍``有限度的''异常 \citeinbeamer{Yu}{TOCS}{02}
	\item[\textcolor{red}{可调节:}] 
      \begin{itemize}
        \item 不同应用对一致性需求不同 \citeinbeamer{Terry}{CACM}{13}
        \item 运行时决定 \citeinbeamer{Terry}{SOSP\&TR}{13}
      \end{itemize}
  \end{description}
\end{frame}
%%%%%%%%%%%%%%%
\begin{frame}{RVSI 定义}
  RVSI {\small (Relaxed Version Snapshot Isolation)} 定义原则:
  \begin{itemize}
    \item<1-> 参数 $k_1, k_2, k_3$ 调节/控制相对于 SI 的``异常''程度
    \item<2-> $\text{RC} \supset \text{RVSI}(k_1, k_2, k_3) \supset \text{SI}$
    \item<2-> $\text{RVSI}(\infty,\infty,\infty) = \text{RC}; \qquad \text{RVSI}(1,0,\ast) = \text{SI}$
  \end{itemize}

  % \vspace{0.60cm}

  % \uncover<1->{
  % \begin{description}
  %   \item[RC {\small (Read Committed)}:] 读取成功提交的更新
  %   \item[SI {\small (Snapshot Isolation)}:] 观察到事务开始之前的最新的系统快照 + 无写冲突事务
  % \end{description}
  % }
\end{frame}
%%%%%%%%%%%%%%%
\begin{frame}{RVSI 定义}
  SI: ``观察到事务开始 \textcolor{blue}{\scriptsize (2) }\textcolor{red}{之前的} 
	\textcolor{blue}{\scriptsize (1) }\textcolor{red}{最新的}系统 \textcolor{blue}{\scriptsize (3) }\textcolor{red}{快照}''
  \begin{cdef}[RVSI {\small (弱化 SI 的要点)}]
	\vspace{5pt}

    \begin{description}
      \item[单变量读 $\texttt{read}(x)$:] \hfill 
        \begin{enumerate}
		  \item 允许读 $\le k_1$ 陈旧值 (\konebv{})
		  \item 允许读 $\le k_2$ 并发更新 (\ktwofv{})
        \end{enumerate}
      \item[多变量读 $\texttt{read}(x), \texttt{read}{(y)}$:] \hfill
        \begin{enumerate}
          \setcounter{enumi}{2}
		\item $\textsl{dist}(\textsl{snap}{(x)},\textsl{snap}{(y)}) \le k_3$ (\kthreesv{})
        \end{enumerate}
    \end{description}
  \end{cdef}
\end{frame}
%%%%%%%%%%%%%%%
\begin{frame}{\chameleon{}~\footnotemark[1]分布式事务键值存储原型系统设计}
  \begin{description}
	\item[系统架构:] 阿里云~\footnotemark[2]多数据中心 {\small ($9 = 3 \times 3$)} % ~\footnotemark[2]\footnotetext[2]{阿里云: \url{https://www.aliyun.com/}.} 
	\item<2->[数据分区:] 同一数据中心
	\item<2->[数据副本:] 跨数据中心; 主从结构
  \end{description}

  \footnotetext[1]{\url{https://github.com/hengxin/chameleon-transactional-kvstore}}
  \footnotetext[2]{\url{https://www.aliyun.com/}}

  \fignocaption{width = 0.55\textwidth}{figures/chameleon-arch.pdf}
\end{frame}
%%%%%%%%%%%%%%%
\begin{frame}{RVSI 维护算法}
  \[
    \textcolor{blue}{\text{RC} \supset \text{RVSI}(k_1, k_2, k_3) \supset \text{SI}}
  \]

  \vspace{0.10cm}

  RVSI 维护算法:
  \begin{itemize}
    \item 以分布式 RC 和 SI 协议 为基础
	  \pause
    \item 事务提交前, 计算 RVSI ``版本约束'' ($k_1, k_2, k_3$ 相关不等式)
	  \begin{description}
		\item[\konebv{}:] $\mpord{x}(\tsts{T_i}) - \mpord{x}(\tcts{T_j}) < k_1$
		\item[\ktwofv{}:] $\mpord{x}(\tcts{T_j}) - \mpord{x}(\tsts{T_i}) \le k_2$
		\item[\kthreesv{}:] $\mpord{x}(\tcts{T_l}) - \mpord{x}(\tcts{T_j}) \le k_3$
	  \end{description}
    \item 事务提交时, 检查 RVSI ``版本约束''
  \end{itemize}

  % \fignocaption{width = 0.50\textwidth}{figures/chameleon-build-passing.png}
  % \textcolor{red}{\small \url{https://github.com/hengxin/chameleon-transactional-kvstore}}
\end{frame}
%%%%%%%%%%%%%%%
\begin{frame}{RVSI 维护算法}
  \begin{description}
	% \item[系统组件:] 客户端库 + 数据中心
	\item[数据分区:] 分布式事务原子提交协议 {\small (2PC)}
	\item[数据副本:] 懒惰复制 {\small (lazy replication)} 协议
  \end{description}

  \fignocaption{width = 0.42\textwidth}{figures/chameleon-framework.pdf}
\end{frame}
%%%%%%%%%%%%%%%
\begin{frame}{RVSI 实验评估}
  \begin{table}[]
  \renewcommand{\arraystretch}{1.1}
  \centering
  \caption{事务负载参数表.\protected\\(\textcolor{blue}
	{评估目标: RVSI 对事务中止率的影响})}
  \resizebox{\textwidth}{!}{%
  \begin{tabular}{|c||c|c|c|}
	\hline
	{\bfseries Parameter}   & {\bfseries F(ixed)/V(ariable)/R(andom)}	
	& {\bfseries Value}		& {\bfseries Explanation}
	\\ \hline  \hline
	\#keys  				& F		& 5  				&  	size of keyspace
	\\ \hline
	\cellcolor{brown}mpl	& \textcolor{red}{V}		& 5, 10, 15, 20, 25, 30
	& \innercell{c}{multiprogramming level: \\ number of concurrent clients} \\ \hline
	\#txs/client					& F		& 1000 						
	& \innercell{c}{number of txs per client}
	\\ \hline
	\#ops/tx					& R		& $\sim$ Binomial(20, 0.5)	
	&  \innercell{c}{number of operations per tx}
	\\ \hline
	\cellcolor{brown}rwRatio & \textcolor{red}{V} 
	  & 1:2, 1:1, 4:1 & {\#reads}/{\#writes}
	\\ \hline
	zipfExponent			& F		& 1		& parameter for Zipfian distribution
	\\ \hline  \hline
	\cellcolor{brown}$(k_1, k_2, k_3)$		& \textcolor{red}{V}
		&  \innercell{c}{(1,0,0) \\ (1,0,2) (1,1,0) \\ (2,0,0) (2,1,2) (2,2,1)}	
		&  for \konebv{}, \ktwofv{}, and \kthreesv{}
	\\ \hline  \hline
	minInterval				& F		& 0ms		& minimum inter-transactions time
	\\ \hline
	maxInterval				& F		& 10ms		& maximum inter-transactions time
	\\ \hline
	meanInterval			& R		& 5ms		
	& \innercell{c}{mean inter-transactions time \\ for exponential distribution}
	\\ \hline
  \end{tabular}
  }
\end{table}

\end{frame}
%%%%%%%%%%%%%%%
\begin{frame}{RVSI 实验评估}
  \fig{width = 0.70\textwidth}{figures/rvsi-rw4-abort-rates.pdf}
  {\textcolor{teal}{读频繁} (rwRatio = 4:1) 负载下 RVSI 对事务中止率的影响.}

	\pause
  \begin{description}
	\item[\textcolor{blue}{wcf-aborted:}] 无显著变化 {\small ($\text{mpl} = 30\text{ 时}, \text{wcf}(1,0,0) = 0.184733$)}
	  \pause
	\item[\textcolor{red}{vc-aborted:}] 显著减少 {\small ($\text{mpl} = 30 \text{ 时}, 
	  \text{vc}(1,0,0) = 0.204733; \text{vc}(1,1,0) = 0.066433; \text{vc}(2,2,1) = 0.002033$)}
  \end{description}
\end{frame}
%%%%%%%%%%%%%%%
\begin{frame}{RVSI 实验评估}
  \begin{columns}
	\column{0.48\textwidth}
	  \fig{width = 1.00\textwidth}{figures/rvsi-rw05-abort-rates.pdf}
		{写频繁 (rwRatio = 1:2) 负载下 RVSI 对事务中止率的影响.}
	\column{0.48\textwidth}
	  \fig{width = 1.00\textwidth}{figures/rvsi-rw1-abort-rates.pdf}
		{读写相当 (rwRatio = 1:1) 负载下 RVSI 对事务中止率的影响.}
  \end{columns}

  \begin{description}
	\item[\textcolor{blue}{wcf-aborted:}] 无显著变化
	\item[\textcolor{red}{vc-aborted:}] 绝对数值小; 相对变化显著 
  \end{description}
\end{frame}
%%%%%%%%%%%%%%%
\begin{frame}{RVSI 的意义}
  \mdf{red}{blue}{RVSI 对事务中止率的影响}{teal}{
	\begin{enumerate}
	  \item 适当放松事务对 RVSI 版本规约的要求可降低事务中止率
	  \item RVSI 能否``显著''降低事务中止率与负载类型相关
    \end{enumerate}
  }
\end{frame}
%%%%%%%%%%%%%%%

%%%%%%%%%%%%%%%

% %%%%%%%%%%%%%%%%%%%%%%%%%%%%%%%%%%%%%%%%%%%%%%%%%%%%%%%%%%%%%%%%%%%%%%%%%%%%%%%	
\section{总结展望}

%%%%%%%%%%%%%%%
\begin{frame}{工作总结}
  \mdf{red}{red}{论文研究问题}{teal}{如何在分布式系统中\\落实\idea{}的\\分布数据一致性问题研究理念?}

  \pause

  \mdf{blue}{blue}{论文主要贡献}{black}{
	\begin{description}
	  \item<2->[理念:] 提出\idea{}的数据\\一致性问题研究理念
	  \item<3->[框架:] 提出包含``一个基础、三个维度''的技术框架 
	  \item<4->[VPC:] 验证 Pipelined-RAM Consistency \hfill \textcolor{brown}{\scriptsize [``精细化, 可度量'']}
	  \item<4->[PA2AM:]	量化 2-Atomicity 协议 \hfill \textcolor{brown}{\scriptsize [``精细化, 可度量'']} 
	  \item<4->[RVSI:] 可调节 Snapshot Isolation \hfill \textcolor{brown}{\scriptsize [``多样化, 可调节'']}
	\end{description}
  }
\end{frame}
%%%%%%%%%%%%%%%
\begin{frame}{未来工作}
  落实\idea{}的数据一致性问题研究理念:

  \pause
  \vspace{0.30cm}

  \begin{enumerate}
	\setlength{\itemsep}{8pt}
	\item VHC {\small (HC: Hybrid Consistency)} 问题 \citeinbeamer{Attiya}{SICOMP}{98} \textcolor{brown}{\small (遵循 VPC 工作思路)}
	  \pause
	\item ``多写模型''下的 atomic 寄存器 \textcolor{brown}{\small (扩展 PA2AM 工作)}
      \begin{description}
		\setlength{\itemsep}{3pt}
        \item[低延迟:] 读操作只需一轮网络通信
        \item[可能性问题:] 是否存在低延迟 ($k$-)atomicity 算法?
		\item[尽可能强:] 如何定义并量化 $p$AM ($p$: probabilistic)?
      \end{description}
	  \pause
	\item 考虑更丰富的数据类型 {\small (replicated data types)} \citeinbeamer{Burckhardt}{POPL}{14}
  \end{enumerate}
\end{frame}
%%%%%%%%%%%%%%%
\begin{frame}{发表论文}

  {\small
  \begin{itemize}
	\item \textcolor{blue}{[TC'16]} {\bf Hengfeng Wei}, Yu Huang, Jian Lu. 
	  Probabilistically-Atomic 2-Atomicity: Enabling Almost Strong Consistency in Distributed Storage Systems. 
	  In {\it IEEE Trans. Comput.}, xx(x):x--x, PrePrints, 2016.
	\item \textcolor{blue}{[TPDS'16]} {\bf Hengfeng Wei}, Marzio De Biasi, Yu Huang, Jiannong Cao, Jian Lu. 
	  Verifying Pipelined-RAM Consistency over Read/Write Traces of Data Replicas.
	  In {\it IEEE Trans. Parallel Distrib. Syst.}, 27(5):1511--1523, 2016.
	\item \textcolor{blue}{[PerCom'12]} {\bf Hengfeng Wei}, Yu Huang, Jiannong Cao, Xiaoxing Ma, Jian Lu. 
	  Formal Specification and Runtime Detection of Temporal Properties for Asynchronous Context. 
	  In {\it Proceedings of the 10th IEEE International Conference on Pervasive Computing and Communications},
	  pages 30--38, 2012.
	\item \textcolor{blue!50!gray}{[WiP for VLDB'17]} {\bf Hengfeng Wei}, Yu Huang, Jian Lu.
	  Relaxed Version Snapshot Isolation in Distributed Transactional Key-Value Stores. 
  \end{itemize}
  }
\end{frame}
%%%%%%%%%%%%%%%
\begin{frame}{致谢}
  \begin{itemize}
	\setlength{\itemsep}{15pt}
	\item 导师: 吕建教授、黄宇副教授
	\item 论文评审与答辩老师
	\item 答辩秘书: 余萍副教授
	\item 软件所全体师生
  \end{itemize}
\end{frame}
%%%%%%%%%%%%%%%
\begin{frame}[noframenumbering]
  \fignocaption{width = 0.20\textwidth}{figures/qa.png}
  \vspace{-0.8cm}
  \begin{center}
    \textcolor{blue}{\bf \large hengxin0912@gmail.com}
  \end{center}
  \vspace{-0.5cm}
  \fignocaption{width = 0.50\textwidth}{figures/thankyou.jpg}
\end{frame}
%%%%%%%%%%%%%%%

% %%%%%%%%%%%%%%%%%%
\appendix

%%%%%%%%%%%%%%%%%%
\section{附录: 参考文献}
%%%%%%%%%%%%%%%%%%
\section{附录: 相关工作}
\hypertarget{appendix}{}

%%%%%%%%%%%%%
%%%%%%%%%%%%%%%
\begin{frame}[label = related-work-categories]{相关工作分类}
  \begin{table}[]
	\centering
	\caption{\idea{}研究理念相关工作.}
	\renewcommand\arraystretch{2}
	\resizebox{\textwidth}{!}{%
	  \begin{tabular}{cc|c|c|c|c|}
		\cline{3-6}
		\multicolumn{2}{c|}{} & \multicolumn{2}{c|}{\textbf{读写寄存器}} & \multicolumn{2}{c|}{\textbf{事务}} \\ \cline{3-6}
		\multicolumn{2}{c|}{} & 多处理器系统 & 分布式系统 & \only<1>{多处理器系统}\only<2>{\textcolor{gray}{多处理器系统}} & 分布式系统 \\ \hline
		\multicolumn{2}{|c|}{\textbf{\ideadt{}}} &  &  
		& \uncover<2>{\multirow{3}{*}{\begin{tabular}[c]{@{}c@{}}\textcolor{gray}{软件}\\ \textcolor{gray}{事务内存}\end{tabular}}} &  
		% \\ \hline
		\\ \cline{1-4} \cline{6-6}
		\multicolumn{1}{|c|}{\multirow{2}{*}{\textbf{\idearm{}}}} & 验证 &  &  &  &  
		% \\ \cline{2-6}
		\\ \cline{2-4} \cline{6-6}
		\multicolumn{1}{|c|}{} & 量化 &  &  &  &  
		\\ \hline
	  \end{tabular}
    }
  \end{table}
\end{frame}
%%%%%%%%%%%%%%%
\begin{frame}{\ideadt{}的研究理念 (一)}
  \begin{table}[]
	\centering
	\renewcommand\arraystretch{1.5}
	\resizebox{\textwidth}{!}{%
	  \begin{tabular}{cc|c|c|c|c|}
		\cline{3-6}
		\multicolumn{2}{c|}{} & \multicolumn{2}{c|}{\textbf{读写寄存器}} & \multicolumn{2}{c|}{\textbf{事务}} \\ \cline{3-6} 
		\multicolumn{2}{c|}{} & 多处理器系统 & 分布式系统 & 多处理器系统 & 分布式系统 \\ \hline
		\multicolumn{2}{|c|}{\textbf{\ideadt{}}} & \only<1-3>{\textcolor{red}{\bf \checkmark}} 
		\only<4->{\begin{tabular}[c]{@{}c@{}}\textcolor{red}{相关工作丰富}\\ \textcolor{red}{理论扎实}\end{tabular}} &  &  &  \\ \hline
	  \end{tabular}%
	}
  \end{table}

  \pause 

  \begin{description}
	\setlength{\itemsep}{5pt}
	\item[典型:] Hybrid consistency \citeinbeamer{Attiya}{SICOMP}{98}
	\item[思想:] 将操作分为强弱两类
	  \pause
	\item[其它:] ``带同步的''一致性模型 \citeinbeamer{Dubois}{IEEE Computer}{88} \citeinbeamer{Steinke}{JACM}{04}
	\item[特点:] 强调正确性 {\footnotesize (properly synchronized)}
  \end{description}
\end{frame}
%%%%%%%%%%%%%%%
\begin{frame}{\ideadt{}的研究理念 (二)}
  \begin{table}[]
	\centering
	\renewcommand\arraystretch{1.5}
	\resizebox{\textwidth}{!}{%
	  \begin{tabular}{cc|c|c|c|c|}
		\cline{3-6}
		\multicolumn{2}{c|}{} & \multicolumn{2}{c|}{\textbf{读写寄存器}} & \multicolumn{2}{c|}{\textbf{事务}} \\ \cline{3-6} 
		\multicolumn{2}{c|}{} & 多处理器系统 & 分布式系统 & 多处理器系统 & 分布式系统 \\ \hline
		\multicolumn{2}{|c|}{\textbf{\ideadt{}}} & {\begin{tabular}[c]{@{}c@{}}{\small 相关工作丰富}\\ {\small 理论扎实}\end{tabular}}
		& \only<1-3>{\textcolor{red}{\bf \checkmark}} 
		\only<4->{\begin{tabular}[c]{@{}c@{}}\textcolor{red}{渐成趋势}\\ \textcolor{red}{理论欠缺}\end{tabular}} &  &  \\ \hline
	  \end{tabular}%
	}
  \end{table}

  \pause 

  \begin{description}
	\setlength{\itemsep}{5pt}
	\item[思想:] 借鉴并发展 Hybrid consistency 的思想
	\item[典型:] 
	  \begin{itemize}
		\item Causal+forced+immediate operations \citeinbeamer{Ladin}{TOCS}{92}
		\item RedBlue consistency \citeinbeamer{Li}{OSDI}{12}
		% \item Apache Cassandra~\footnote{\url{http://cassandra.apache.org/}} \citeinbeamer{Facebook}{SIGOPS OSR}{10}
		\item Pileus \citeinbeamer{Terry}{SOSP}{13}
	  \end{itemize}
	  \pause
	\item[特点:] 更细粒度的多一致性模型共存、更能容忍数据不一致
  \end{description}
\end{frame}
%%%%%%%%%%%%%%%
\begin{frame}{\ideadt{}的研究理念 (三)}
  \begin{table}[]
	\centering
	\renewcommand\arraystretch{1.5}
	\resizebox{\textwidth}{!}{%
	  \begin{tabular}{cc|c|c|c|c|}
		\cline{3-6}
		\multicolumn{2}{c|}{} & \multicolumn{2}{c|}{\textbf{读写寄存器}} & \multicolumn{2}{c|}{\textbf{事务}} \\ \cline{3-6} 
		\multicolumn{2}{c|}{} & 多处理器系统 & 分布式系统 & 多处理器系统 & 分布式系统 \\ \hline
		\multicolumn{2}{|c|}{\textbf{\ideadt{}}} & {\begin{tabular}[c]{@{}c@{}}{\small 相关工作丰富}\\ {\small 理论扎实}\end{tabular}}
		& \begin{tabular}[c]{@{}c@{}}{\small 渐成趋势}\\ {\small 理论欠缺}\end{tabular} 
		& \textcolor{gray}{\small 软件事务内存}
		& \only<1-2>{\textcolor{red}{\bf \checkmark}} 
		  \only<3->{\textcolor{red}{探索阶段}} \\ \hline
		  % \only<4->{\begin{tabular}[c]{@{}c@{}}\textcolor{red}{探索阶段}\\ \textcolor{red}{系统、理论欠缺}\end{tabular}} \\ \hline
	  \end{tabular}
	}
  \end{table}

  \begin{description}
	\setlength{\itemsep}{5pt}
	\item[思想:] 多个事务一致性模型共存
	\item[典型:] 
	  \begin{itemize}
		\item RC-SR {\scriptsize (relaxed currency serializability)} \citeinbeamer{Bernstein}{SIGMOD}{06}
		\item Pileus consistency choices \citeinbeamer{Terry}{MSR-TR}{13}
		\item Multi-level CSI {\scriptsize (Causal Snapshot Isolation)} \citeinbeamer{Tripathi}{BigData}{15}
	  \end{itemize}
	  \pause
	\item[挑战:] ``多样化''事务语义; 可扩展的系统实现
  \end{description}
\end{frame}
%%%%%%%%%%%%%%%
\begin{frame}{\idearm{}的研究理念 (一)}
  \begin{table}[]
	\centering
	\renewcommand\arraystretch{1.6}
	\resizebox{\textwidth}{!}{%
	  \begin{tabular}{cc|c|c|c|c|}
		\cline{3-6}
		\multicolumn{2}{c|}{} & \multicolumn{2}{c|}{\textbf{读写寄存器}} & \multicolumn{2}{c|}{\textbf{事务}} \\ \cline{3-6} 
		\multicolumn{2}{c|}{} & 多处理器系统 & 分布式系统 & 多处理器系统 & 分布式系统 \\ \hline
		\multicolumn{1}{|c|}{\only<1-2>{\multirow{2}{*}{\textbf{\idearm{}}}}\only<3>{\multirow{2}{*}[-0.5em]{\textbf{\idearm{}}}}\only<4->{\multirow{2}{*}[-1em]{\textbf{\idearm{}}}}} 
		& 验证 
		& \only<2>{\textcolor{red}{\bf \checkmark}}\only<3>{\begin{tabular}[c]{@{}c@{}}\textcolor{red}{典型模型}\\ \textcolor{red}{理论全面}\end{tabular}}\only<4>{\begin{tabular}[c]{@{}c@{}}{\small 典型模型}\\ {\small 理论全面}\end{tabular}} &  &  &  \\ \cline{2-6} 
		\multicolumn{1}{|c|}{} & 量化 & \only<4->{\begin{tabular}[c]{@{}c@{}}\textcolor{red}{暂无}\\ \textcolor{red}{强调正确性}\end{tabular}} &  &  &  \\ \hline
	  \end{tabular}
	}
  \end{table}

  \pause

  典型的一致性模型验证 \textcolor{blue}{\small (Verify)} 问题:
  \begin{itemize}
	\item VSC {\scriptsize (Sequential Consistency)}, VL {\scriptsize (Linearizability)} \citeinbeamer{Gibbons}{SICOMP}{97}
	\item VMC {\scriptsize (Memory Coherence)} \citeinbeamer{Cantin}{TPDS}{05}
	\item VTSO {\scriptsize (Total Store Order)} \citeinbeamer{Hangal}{ISCA}{04} \citeinbeamer{Manovit}{SPAA}{05} 
	  \citeinbeamer{Roy}{CAV}{06} \citeinbeamer{Baswana}{CAV}{08}
  \end{itemize}
\end{frame}
%%%%%%%%%%%%%%%
\begin{frame}{\idearm{}的研究理念 (二)}
  \begin{table}[]
	\centering
	\renewcommand\arraystretch{1.6}
	\resizebox{\textwidth}{!}{%
	  \begin{tabular}{cc|c|c|c|c|}
		\cline{3-6}
		\multicolumn{2}{c|}{} & \multicolumn{2}{c|}{\textbf{读写寄存器}} & \multicolumn{2}{c|}{\textbf{事务}} \\ \cline{3-6} 
		\multicolumn{2}{c|}{} & 多处理器系统 & 分布式系统 & 多处理器系统 & 分布式系统 \\ \hline
		\multicolumn{1}{|c|}{\multirow{2}{*}[-1em]{\textbf{\idearm{}}}} & 验证 
		& \begin{tabular}[c]{@{}c@{}}{\small 典型模型}\\ {\small 理论全面}\end{tabular} 
		& \only<1-2>{\textcolor{red}{\bf \checkmark}}\only<3->{\begin{tabular}[c]{@{}c@{}}\textcolor{red}{弱模型验证}\\ \textcolor{red}{有待研究}\end{tabular}} &  &  \\ \cline{2-6} 
		\multicolumn{1}{|c|}{} & 量化 & \begin{tabular}[c]{@{}c@{}}{\small 暂无}\\ {\small 强调正确性}\end{tabular} &  &  &  \\ \hline
	  \end{tabular}
	}
  \end{table}

  \begin{description}
	\item[动机:] 商业条款 SLA {\scriptsize (Service Level Agreement)} \citeinbeamer{Amazon}{SOSP}{07}
	  \pause
	\item[特点:] 在线验证 safeness, regularity, atomicity \citeinbeamer{Golab}{PODC}{11}
	  \pause
	\item[不足:] 常用 Pipelined-RAM consistency, causal consistency, \\hybrid consistency 验证问题有待研究
  \end{description}
\end{frame}
%%%%%%%%%%%%%%%
\begin{frame}{\idearm{}的研究理念 (三)}
  \begin{table}[]
	\centering
	\renewcommand\arraystretch{1.6}
	\resizebox{\textwidth}{!}{%
	  \begin{tabular}{cc|c|c|c|c|}
		\cline{3-6}
		\multicolumn{2}{c|}{} & \multicolumn{2}{c|}{\textbf{读写寄存器}} & \multicolumn{2}{c|}{\textbf{事务}} \\ \cline{3-6} 
		\multicolumn{2}{c|}{} & 多处理器系统 & 分布式系统 & 多处理器系统 & 分布式系统 \\ \hline
		\multicolumn{1}{|c|}{\multirow{2}{*}[-1em]{\textbf{\idearm{}}}} & 验证 
		& \begin{tabular}[c]{@{}c@{}}{\small 典型模型}\\ {\small 理论全面}\end{tabular} 
		& \begin{tabular}[c]{@{}c@{}}{\small 弱模型验证}\\ {\small 有待研究}\end{tabular} 
		&  &  \\ \cline{2-6} 
		\multicolumn{1}{|c|}{} & 量化 
		& \begin{tabular}[c]{@{}c@{}}{\small 暂无}\\ {\small 强调正确性}\end{tabular}
		& \only<2>{\textcolor{red}{\bf \checkmark}}\only<3>{\begin{tabular}[c]{@{}c@{}}\textcolor{red}{量化执行易}\\ \textcolor{red}{量化协议难}\end{tabular}}
		&  &  \\ \hline
	  \end{tabular}
	}
  \end{table}

  \pause

  \begin{description}
	\item[量化执行:] $k$/$\Delta$/$\Gamma$-atomicity {\tiny{\textcolor{blue}{[Golab@PODC'11, ICDCS'13, ICDCS'14, PODC'15]}}}
	\item[量化协议:] probabilistic regularity/atomicity \\ \citeinbeamer{Yu}{DISC}{03} \citeinbeamer{Lee}{DC}{05} \citeinbeamer{Gramoli}{OPODIS}{07} \citeinbeamer{Bailis}{PVLDB}{12}
  \end{description}
\end{frame}
%%%%%%%%%%%%%%%
\begin{frame}{\idearm{}的研究理念 (四)}
  \begin{table}[]
	\centering
	\renewcommand\arraystretch{1.6}
	\resizebox{\textwidth}{!}{%
	  \begin{tabular}{cc|c|c|c|c|}
		\cline{3-6}
		\multicolumn{2}{c|}{} & \multicolumn{2}{c|}{\textbf{读写寄存器}} & \multicolumn{2}{c|}{\textbf{事务}} \\ \cline{3-6} 
		\multicolumn{2}{c|}{} & 多处理器系统 & 分布式系统 & \textcolor{gray}{多处理器系统} & 分布式系统 \\ \hline
		\multicolumn{1}{|c|}{\multirow{2}{*}[-1em]{\textbf{\idearm{}}}} & 验证 
		& \begin{tabular}[c]{@{}c@{}}{\small 典型模型}\\ {\small 理论全面}\end{tabular} 
		& \begin{tabular}[c]{@{}c@{}}{\small 弱模型验证}\\ {\small 有待研究}\end{tabular} 
		& \multirow{2}{*}[-1em]{\begin{tabular}[c]{@{}c@{}}\textcolor{gray}{\small 软件事务内存}\end{tabular}}
		& \only<1-2>{\textcolor{red}{\bf \checkmark}}\only<3->{\begin{tabular}[c]{@{}c@{}}\textcolor{red}{理论全面}\\ \textcolor{red}{指导协议设计}\end{tabular}}
		\\ \cline{2-4} \cline{6-6}
		\multicolumn{1}{|c|}{} & 量化 
		& \begin{tabular}[c]{@{}c@{}}{\small 暂无}\\ {\small 强调正确性}\end{tabular}
		& \begin{tabular}[c]{@{}c@{}}{\small 量化执行易}\\ {\small 量化协议难}\end{tabular}
		&  &  \\ \hline
	  \end{tabular}
	}
  \end{table}

  \begin{itemize}
	\item SR {\scriptsize (Serializability)} 强一致性模型及变体  
	  \\ \citeinbeamer{Papadimitriou}{JACM}{79} \citeinbeamer{Bernstein}{TODS}{83} \citeinbeamer{Yannakakis}{JACM}{84} 
	\item SI {\scriptsize (Snapshot Isolation)} 等弱一致性模型 
	  \\ \citeinbeamer{Adya}{Phd-Thesis}{99} \citeinbeamer{Fekete}{TODS}{05} \citeinbeamer{Cahill}{SIGMOD}{08} \citeinbeamer{Zellag}{VLDB}{14}
  \end{itemize}
\end{frame}
%%%%%%%%%%%%%%%
\begin{frame}{\idearm{}的研究理念 (五)}
  \begin{table}[]
	\centering
	\renewcommand\arraystretch{1.6}
	\resizebox{\textwidth}{!}{%
	  \begin{tabular}{cc|c|c|c|c|}
		\cline{3-6}
		\multicolumn{2}{c|}{} & \multicolumn{2}{c|}{\textbf{读写寄存器}} & \multicolumn{2}{c|}{\textbf{事务}} \\ \cline{3-6} 
		\multicolumn{2}{c|}{} & 多处理器系统 & 分布式系统 & \textcolor{gray}{多处理器系统} & 分布式系统 \\ \hline
		\multicolumn{1}{|c|}{\multirow{2}{*}[-1em]{\textbf{\idearm{}}}} & 验证 
		& \begin{tabular}[c]{@{}c@{}}{\small 典型模型}\\ {\small 理论全面}\end{tabular} 
		& \begin{tabular}[c]{@{}c@{}}{\small 弱模型验证}\\ {\small 有待研究}\end{tabular} 
		& \multirow{2}{*}[-1em]{\begin{tabular}[c]{@{}c@{}}\textcolor{gray}{\small 软件事务内存}\end{tabular}}
		& \begin{tabular}[c]{@{}c@{}}{\small 理论全面}\\ {\small 指导协议设计}\end{tabular}
		\\ \cline{2-4} \cline{6-6}
		\multicolumn{1}{|c|}{} & 量化 
		& \begin{tabular}[c]{@{}c@{}}{\small 暂无}\\ {\small 强调正确性}\end{tabular}
		& \begin{tabular}[c]{@{}c@{}}{\small 量化执行易}\\ {\small 量化协议难}\end{tabular}
		&  
		& \begin{tabular}[c]{@{}c@{}}\textcolor{red}{量化协议难}\\ \textcolor{red}{相关工作少}\end{tabular} 
		\\ \hline
	  \end{tabular}
	}
  \end{table}

  \begin{itemize}
	\item 量化 SI {\scriptsize (Snapsho Isolation)} 与 RC {\scriptsize (Read Committed)} 协议 \citeinbeamer{Fekete}{PVLDB}{09}
  \end{itemize}
\end{frame}
%%%%%%%%%%%%%%%
\begin{frame}[label = related-work-summary]{相关工作总结}
  \begin{table}[]
	\centering
	\caption{\idea{}研究理念相关工作.}
	\renewcommand\arraystretch{1.6}
	\resizebox{\textwidth}{!}{%
	  \begin{tabular}{cc|c|c|c|c|}
		\cline{3-6}
		\multicolumn{2}{c|}{} & \multicolumn{2}{c|}{\textbf{读写寄存器}} & \multicolumn{2}{c|}{\textbf{事务}} \\ \cline{3-6} 
		\multicolumn{2}{c|}{} & 多处理器系统 & \cellcolor{red!80}{分布式系统} & \textcolor{gray}{多处理器系统} & \cellcolor{red!80}{分布式系统} \\ \hline

		\multicolumn{2}{|c|}{\textbf{\ideadt{}}} & {\begin{tabular}[c]{@{}c@{}}{\small 相关工作丰富}\\ {\small 理论扎实}\end{tabular}}
		& \begin{tabular}[c]{@{}c@{}}{\small 渐成趋势}\\ {\small 理论欠缺}\end{tabular} 
		& \multirow{2}{*}[-3em]{\begin{tabular}[c]{@{}c@{}}\textcolor{gray}{\small 软件}\\ \textcolor{gray}{\small 事务内存}\end{tabular}}
		& \cellcolor{brown!80}{\small 探索阶段} \\ \cline{1-4} \cline{6-6}

		\multicolumn{1}{|c|}{\multirow{2}{*}[-1em]{\textbf{\idearm{}}}} & 验证 
		& \begin{tabular}[c]{@{}c@{}}{\small 典型模型}\\ {\small 理论全面}\end{tabular} 
		& \cellcolor{brown!80}{\begin{tabular}[c]{@{}c@{}}{\small 弱模型验证}\\ {\small 有待研究}\end{tabular}}
		& 
		& \begin{tabular}[c]{@{}c@{}}{\small 理论全面}\\ {\small 指导协议设计}\end{tabular}
		\\ \cline{2-4} \cline{6-6} \hhline{*{3}{~}-*{2}{~}}

		\multicolumn{1}{|c|}{} & 量化 
		& \begin{tabular}[c]{@{}c@{}}{\small 暂无}\\ {\small 强调正确性}\end{tabular}
		& \cellcolor{brown!80}{\begin{tabular}[c]{@{}c@{}}{\small 量化执行易}\\ {\small 量化协议难}\end{tabular}}
		&  
		& \begin{tabular}[c]{@{}c@{}}{\small 量化协议难}\\ {\small 相关工作少}\end{tabular} 
		\\ \hline
	  \end{tabular}
	}
  \end{table}
\end{frame}
%%%%%%%%%%%%%%%

%%%%%%%%%%%%%
% \begin{frame}[label = related-work-appendix]
%   {\hyperlink{related-work-main}{\beamerbutton{}}}
% \end{frame}
%%%%%%%%%%%%%


\end{document}